% Created 2022-12-27 mar 00:56
% Intended LaTeX compiler: pdflatex
\documentclass[letterpaper]{book}
\usepackage[spanish, activeacute, english]{babel}
\usepackage{amsmath, amsthm, amssymb, amsfonts}
\usepackage[table, xcdraw]{xcolor}
\usepackage[utf8]{inputenc}
\usepackage[normalem]{ulem}
\usepackage[T1]{fontenc}
\usepackage{graphicx}
\usepackage{grffile}
\usepackage{longtable}
\usepackage{wrapfig}
\usepackage{rotating}
\usepackage{textcomp}
\usepackage{capt-of}
\usepackage{hyperref}
\usepackage{graphicx}
\usepackage{pst-node}
\usepackage{tikz-cd}
\usepackage{float}
\usepackage{amscd}
\usepackage{color}
\usepackage{transparent}
\usepackage{makeidx}
\usepackage{afterpage}
\usepackage{array}
\graphicspath{{./figs/}}

\newtheorem{theorem}{Theorem}[section]
\newtheorem{prop}[theorem]{Proposition}
\newtheorem{cor}[theorem]{Corollary}
\newtheorem{lema}[theorem]{Lema}
\newtheorem{definition}{Definition}[section]
\newtheorem{conjeture}{Conjeture}
\newtheorem{example}{Example}

\renewcommand{\figurename}{Figure}
\newcommand{\zah}{\ensuremath{ \mathbb Z }}
\newcommand{\nat}{\ensuremath{ \mathbb N }}
%%\newcommand{\dem}{{\sc Demostraci\'on. }}
\newcommand{\bg}{\ensuremath{\overline \Gamma}}
\newcommand{\ga}{\ensuremath{\Gamma}}
\newcommand{\fb}{\ensuremath{\overline f}}
\newcommand{\la}{\ensuremath{\lambda}}
\newcommand{\La}{\ensuremath{\Lambda}}
\newcommand{\bt}{\ensuremath{\overline T}}
\newcommand{\li}{\ensuremath{\mathbb{L}}}
\newcommand{\ord}{\ensuremath{\mathbb{O}}}
\newcommand{\co}{\ensuremath{\mathbb C }}
\newcommand{\con}{\ensuremath{\mathbb{C}^n}}
\newcommand{\cp}{\ensuremath{\mathbb{CP}}}
\newcommand{\rp}{\ensuremath{\mathbb{RP}}}
\newcommand{\re}{\ensuremath{\mathbb R }}
\newcommand{\hc}{\ensuremath{\widehat{\mathbb C} }}
\newcommand{\pslz}{\ensuremath{PSL(2,\mathbb Z) }}
\newcommand{\pslr}{\ensuremath{PSL(2,\mathbb R) }}
\newcommand{\pslc}{\ensuremath{PSL(2,\mathbb C) }}
\newcommand{\hd}{\ensuremath{\mathbb H^2}}


\author{Carlos Eduardo Martinez Aguilar}
\title{Tesis: Holomorphic Foliations with bounded volumes in Kähler Manifolds}
\begin{document}

\maketitle
\tableofcontents
\newpage

\chapter{Preliminaries}
Some notation that will be used throughout the thesis: $\con$ will be the vector space of dimension $n$ over $\co$ the set of complex numbers. Given an open set $U\subset\con$ the set of holomorphic functions $f:U\rightarrow\co$ will be denoted as $\mathcal{H(U)}$ and $\mathcal{O}_{U}$ will represent the sheaf of holomorphic germs over $U$.
For a set of functions $\{f_{1},\dots,f_{k}\}\subset\mathcal{O}_{U}$ its vanishing set is denoted and defined as
\[
  V(f_{1},\dots,f_{k}):=\{x\in U\,\vert\,f_{1}=\dots =f_{k}=0\}.
\]
The set of complex lines in a $d$ dimensional complex vector spaces, also known as the complex projective space, will be denoted as $\cp^{d-1}$. Complex and real manifolds will be denoted usually with the letters $L, N, Ñ, M,$. etc. The tanget space of a manifold wil be denoted as $TM$ and the space of smooth sections (vector fields) of $TM$ will be denoted as $\mathcal{T}M$.
An (almost)-complex structure of an even dimenstional real manifold will be denoted as $J:TM\rightarrow TM$, with $J²=-id_{TM}$.Moreover the complexification of a even dimentional real manifold will be denoted as $TM_{\co}:=TM\otimes\co$, and the complex conjugate splitting will be denoted as $TM_{\co}=TM^{1,0}\oplus TM^{0,1}$ of isomorphic vector spaces.
Similarly the space of $(p,q)$-forms on a complex manifold $M$ is denoted as $\Omega^{(p,q)}(M)$, defined as the section space of the vector bundles
\[
  \bigwedge^{p,q}T^{*}M=\bigwedge^{p}T^{*}M\otimes_{\co}\bigwedge^{q}T^{*}M.
\]
\noindent where the following \emph{``conjugation''} convention is asumed: $\overline{\Omega^{(p,q)}(M)}=\Omega^{(q,p)}(M)$, with the space of complex differential forms given as the space
\[
  \Omega^{k}_{\co}(M)=\bigoplus_{p+q=k}\Omega^{p,q}(M).
\]
\noindent The differential operator is defined as $d=\partial + \overline{\partial}$, where $\partial:\Omega^{(p,q)}(M)\rightarrow \Omega^{(p+1,q)}(M)$ is the holomorphic differential operator and $\overline{\partial}:\Omega^{(p,q)}(M)\rightarrow\Omega^{(p,q+1)}(M)$ is the anti-holomorphic differential operator.\\
Unless stated otherwise a submanifold will be understood as an embedded submanifold, i.e. $N$ is an embedded (differential/complex) submanifold of $M$ if $N$ is a (differential/complex) manifold such that there exist $f:N\rightarrow M$ a (differental/holomorphic) immersion.
A foliation $\mathfrak{F}$ on a manifold $M$ of real dimension $n$, can be defined in a numerous of distict and interesting ways, depending on the application needed a foliation will be understood as
\begin{itemize}
  \item A decomposition of $M$ into disjoint connected submanifolds called \emph{leafs},
        \[
          M=\bigcup_{\alpha}\mathcal{L_{\alpha}},
        \]
        \noindent such that for every point $p\in M$ there exists an open set and a ($\mathcal{C}^{k}$, differential, analytic or holomorphic) function $(U,\phi)$, with $\phi\rightarrow\re^{m}$ and such that every $\mathcal{L}_{\alpha}\cap U$ is a level set of $\phi$.
  \item A \emph{foliated chart}, meaning a set $\{(U_{i},\varphi_{i})\}$ such that $\{U_{i}\}$ is a covering of $M$ and $\varphi_{i}:U_{i}\rightarrow\re^{n}$ is a ($\mathcal{C}^{k}$, differential, analytic or holomorphic) chart such that
        \[
          \varphi_{i}(U_{i})=W_{i}\times\re^{m},\text{ with }W_{i}\subset\re^{n-m}.
        \]
  \item An \emph{involutive} distribution of the tangent bundle $TM$, i. e. A $\mathcal{C}^{\infty}(M)$-module $\mathfrak{F}\subset\mathcal{T}M$ closed under the lie bracket $[\cdot,\cdot]$.
  \item The dual poin of view from the previous viewpoint will also be adopted, i. e. as a set of $k$-forms that null over an involutive disribution.
\end{itemize}
If all the leafs are of the same dimension, then the foliation is said to be \emph{regular}, note that a foliated chart determines the dimensionof the leafs. This is equivalent to all the spaces generated by the poinwise evaluations of $\mathcal{F}$ being of the same dimension.
The \emph{holonomy grupoid} also called the \emph{graph} of a foliation $(M,\mathcal{F})$ is the set of all curves cointained in the leaves of the foliation \textit{modulo homotopy} and will be denoted as $G(M,\mathcal{F})$.

\section{Complex manifolds, analytic varieties and their limits}
An \emph{analytic variety} is defined as a paracompact Hausdorff space $X$ with a locally ringed structure $\mathcal{H}_{X}$ modeled by the local ring structure of
a vanishing set in some finite dimensional complex vector space, that is, $(X,\mathcal{H}_{X})$ locally isomorphic to a local model $(Z,\mathcal{O}_{U}/(f_{1},\dots,f_{k}))$, where $U\subset\con$ is an open set and $Z=V(f_{1},\dots,f_{k})$, and $(f_{1},\dots,f_{k})$ denotes the ideal generated by the functions $\{f_{1},\dots,f_{k}\}$. Given a open set $V\subset X$ and an open set $U\subset\co$ and an isomorphism between ringed structures as discussed, $\varphi:U\rightarrow V$, sometimes the pair $(V,\varphi)$ will be called a \emph{chart}.

Moreover it is said that an analytic variety $X$ is of pure dimension $k$ if all of its irreducible components are of the same dimension. Similarly the definition of a complex algebraic variety is the same, only considering the ring of polynomials $\co[z_{1},\dots,z_{n}]$ as its structure ring.

An \emph{analytic subspace} of an analytic variety $(X,\mathcal{H}_{X})$ consist of a subset $Y\subset X$ such that for every $y\in Y$ there is a chart $(U,\varphi)$ around $y$ with
\[
  \varphi(Y\cap U)=V_U(\varphi(\eta_1),\dots,\varphi(\eta_k)),
\]

\noindent and $\lbrace\eta_1,\dots,\eta_k\rbrace\subset\mathcal{H}_{X}$, naturally we have a local ring structure $(Y,\mathcal{H}_{Y})$ given by $\mathcal{H}_Y:=\mathcal{H}_n/\mathcal{I}(Y)$, as before.
Note that using projective coordinates, the set of vanishing points of homogeneous polynomials are well defined on $\cp^n$. Analytic and algebraic subsets of $\cp^n$ are going to be called  \emph{\textbf{projective} analytic subsets} and \emph{\textbf{projective} algebraic subsets} respectively.

\subsection{Kähler manifolds and varieties}
Kähler manifolds are special because of the relationship between their riemannian metrics and their associated symplectic $(1,1)$-forms, one aspect of this relationship is the fact that both both asociated volume forms coincide.
\begin{definition}
        Let $M$ be a complex manifold with with respect to the integrable complex structure
        $J:TM\rightarrow TM$. A \textit{Hermitian metric} on $M$ is a smooth family
        real bilinear forms $\lbrace h_p\rbrace_p$ for all $p\in M$ where
        $h_p:T_pM\times T_pM\rightarrow\co$, such that
\begin{itemize}
        \item $h_p(JX,JY)=h_p(X,Y)\hspace{0.3cm}\forall\lbrace X,Y\rbrace\subset T_pM\hspace{0.3cm}\forall p\in M.$
        \item $h_p(X,JX)>0\hspace{0.3cm}\forall X\in T_p M\setminus\lbrace 0\rbrace\hspace{0.3cm}\forall p\in M.$
        \item $h_p(X,Y)=\overline{h_p(Y,X)}\hspace{0.3cm}\forall\lbrace X,Y\rbrace\subset T_pM\hspace{0.3cm}\forall p\in M.$
\end{itemize}
An Hermitean metric $h$ will also be denoted as $\langle\cdot,\cdot\rangle$, a manifold with a Hermitian metric $(M,h)$ its called a \textit{Hermitian manifold}.
\end{definition}
An Hermitian metric $h$ has an associated Riemannian metric and a associated $(1,1)$-form given by the real and
imaginary part of $h$ respectively. Conversely, a Riemannian manifold of even dimension $(M,g)$ with a complex structure $J$ has
a natural Hermitian metric given by
\[
  h(X,Y):=g(X,Y)-i\omega(X,Y).
\]
\noindent Where $\omega$ is its associated 2-form given by $\omega(X,Y)=g(JX,Y)$ which is clearly antisymmetric since $J^2=-id_M$.
\begin{definition}
        Let $(M,h)$ be a Hermitian manifold with Hermitian metric $h$ and associated $(1,1)$-form $\omega$, $h$ is a Kähler metric if
        \[
          d\omega=0,
        \]
        \noindent an Hermitean manifold with a Kähler metric is called a \textit{Kähler manifold}, and its associated $(1,1)$-form is  called a \textit{Kähler form} or a \textit{Kähler symplectic form}.
\end{definition}
\begin{example}
  A classic example that will be used al the time in this thesis is the complex projetive space of dimesion $n$ $\cp^{n}$ with the Fubini-Study metric, which in homogeneous coordinates $Z=[1:\zeta_{1}\dots:\zeta_{n}]=[1:\zeta]$ is defined by the Kähler form
  \[
    \omega_{FS}=\frac{i}{2}\partial \overline{\partial}\log(|Z|^{2})=\frac{i}{2}\partial \overline{\partial}\log(1+|\zeta|^{2}).
  \]
\end{example}
Kähler manifolds have some strong topological restrictions, for example, powers of the Kähler form $\omega^k$ are non trivial representatives of cohomology classes in $H^{2k}(M ;\re)$, meaning these groups are never trivial. More importantly complex submanifolds of a Kähler manifold are also K\"ahler.
Another important consequence is the following famous result
\begin{theorem}[Wirtinger's inequality]
        Let $M$ be a Kähler manifold with Kähler form $\omega_M$ and let $f:N\rightarrow M$ be a closed oriented immersion
        of an oriented real manifold of real dimension $2k$. Let $\omega=f^*\,\omega_M$, then
        \begin{equation}
                        \frac{\omega^k}{k!}\leq dVol_N\hspace{0.3cm}\text{where }dVol_N\text{ is the volume form of N,}
        \end{equation}
        and the equality holds if and only if $N$ is a complex submanifold of $M$.
\end{theorem}
\subsection{Kähler manifolds and their volumes}
\noindent another way to state Wirtinger's inequality is the following
\begin{theorem}\label{wirtinger}
        Let $N$ be a complex compact submanifold with boundary of a Kähler manifold $M$, then $N$ is a volume minimizing
        submanifold in its homology class $H_{2k}(M,\partial N,\zah),$ meaning that for any real submanifold $X$ of
        real dimension $2k$ and boundary $\partial N$ homologous to $N$ has
        \[
          Vol_{2k}(N)\leq Vol_{2k}(X).
        \]
\end{theorem}
\noindent Let $\omega=-Im\langle\cdot,\cdot\rangle$ be the standard $(1,1)$-form of the standard euclidean Kähler metric in $\con$ and let $M$ be a complex submanifold of $\con$ by Wirtinger's inequality, if $Vol_{2k}(M)$ is the volume of $M$ given by the riemannian structure $g=Re\langle\cdot,\cdot\rangle\vert_{M}$, then
\begin{equation}
        Vol_{2k}(M)= \frac{1}{k!}\int_M \omega^k.
\end{equation}
If $X$ is a purely $k$ dimensional analytic subset of $\con$ and $\Sigma(X)$ is
its singular locus, then $M=X\setminus\Sigma(X)$ is a complex manifold and
since $\Sigma(X)$ is an analytic subset of $X$ of lesser dimension its volume is negible, i.e.
\[
    Vol_{2k}(X)= \frac{1}{k!}\int_{x} \omega^k=
                              \frac{1}{k!}\int_{X\setminus\Sigma(X)} \omega^k=
                                                            Vol_{2k}(X\setminus\Sigma(X)).
\]
\subsection{Volumes of Analytic varieties and Hausdorff limits}
\noindent Besides the volumes of manifolds and analytic sets, a very useful tool that is directly related to the volumes is the Hausdorff measure, which can be conceptualized as a generalization of the volume for arbitrary dimensions and is in itself a tool for measuring dimensions as an arbitrary non-negative real number.
The $\delta$-Hausdorff measure of a subset $S\subset\con$ is denoted by $\mathcal{H}_{\delta}(S)$, here is a list some of the most important properties of this measure, see \cite{Stolzenberg}[ch. 3]

\begin{enumerate}
        \item[1.] If $\mathcal{H}_{\delta}(S)<\infty$, and $\delta<\kappa$, then $\mathcal{H}_{\kappa}(S)=0$.\\

        \item[2.] If $f:X\rightarrow Y$ is a Lipschitz continuous function with
        Lipschitz constant $\lambda$, then for any $\delta\in\re^{+}$ and
        $S\subset X$, the following inequality holds
        \[
        \mathcal{H}_{\delta}(f(S))\leq\lambda^{\delta}\mathcal{H}_{\delta}(S).
        \]
        \item[3.]If $X=\re^{n}$ and $S$ is a smooth submanifold of dimension
        $k\in\zah^{+}$, then the volume of $S$ as a submanifold is related to
        its Hausdorff measure by the formula
        \[
        Vol_{k}(M)=\alpha_{k}\mathcal{H}_k(S),\hspace*{0.3cm}\alpha_k= \frac{\pi^k}{\Gamma(k/2+1)},
        \]
\noindent where $\Gamma(z)$ is the Euler's gamma function.
\end{enumerate}

Related to the Hausdorff measure is the Hausdorff metric defined for $K_1$
and $K_2$ compact subsets of a metric space $(X,d)$ as $$
d_h(K_1,K_2):= \max_{x\in K_1}\left\{d(x,K_2)\right\}+\max_{y\in K_2}\left\{d(y,K_1)\right\}.$$

The Hausdorff metric allows to define a convenient notion of convergence of closed subsets: let $\{S_n \}$ be anysequence of closed subsets of $X$, then $S_n$ converges to the closed subset $S$ in the Hausdorff metric $S_n\overset{h}\rightarrow S$, if for every $K\subset X$ compact we have
\[
  d_h(K\cap S_n,K\cap S)\rightarrow 0.
\]
The first of Bishop's results that is very useful for understanding some of the analytical properties of purely $k$-dimensional subvarieties is the following convergence theorem
\begin{theorem}[Sequence theorem] \label{bishop sequence}Let $\lbrace
        V_n\rbrace$ be a sequence of purely $k$-dimensional subvarieties of a
        domain $\Omega\subset\con$ such that $V_n\overset{h}\rightarrow V$, with
        $V\subset\Omega$ a closed subset. If $$ Vol_{2k}(V_n)\leq M
        \hspace*{0.2cm}\forall n\in\nat, $$
        \noindent for $M\in\re$ a positive constant, then for the Hausdorff
        measure we have $\mathcal{H}_{2k+1}(V)=0$, moreover $V$ is a purely
        $k$-dimensional analytic subvariety of $\Omega$.
\end{theorem}
As a direct application of this result, one can show the following very useful proposition (see \cite{Stolzenberg}[ch. 4]).
\begin{theorem}\label{bishop mapping}[Bishop's proper mapping theorem] Let
        $\Omega\subset\con$ be a domain that contains $0$ and let
        $S\subset\Omega$ be a closed subset. If $\mathcal{H}_{2k+1}(S)=0$, then there
        is a suitable coordinate change of $\con$, $(z_1,\ldots,z_n)$ and
        neighborhoods $\Omega_k\subset\co^k$ and $\Omega_{n-k}\subset\co^{n-k}$, such that
        $0\in\Omega_k\times\Omega_{n-k}\subset\Omega$ and the projection
        \[
          \pi_k:S\cap(\Omega_k\times\Omega_{n-k})\rightarrow\Omega_k\, ,\hspace{0.3cm}\pi_k(z,w):=z,
        \]
        \noindent is a proper map.
\end{theorem}
\noindent This applied to purely $k$-dimensional analytic
subvarieties implies the regular coordinates theorem, meaning the proper
mapping is a finite sheeted analytic covering.

\subsection{Consequences of the proper mapping theorem}

\noindent As mentioned, this result by Bishop can be used to prove many other important
results (see \cite{Stolzenberg}) one of the most significant is the proof of
Remmert-Stein's theorem, this was generalized and proved by Bishop in
\cite{Bishop}.

\begin{theorem}\label{Rem-Stein}[Remmert-Stein]
        Let $\Omega\subset\con$ be an open subset and $Y$ an analytic subset
        of $\Omega$ and let $X$ be a analytic subset of
        $\Omega\setminus Y$. If $Y$ is of dimension at most $k-1$ and $X$ is of pure dimension $k$,
        then the closure of $X$, $\overline{X}\cap\Omega$ is an analytic subset of $\Omega$.
\end{theorem}
This is an essential step towards the proof of Chow's theorem if one is trying to avoid using categorical methods and
quasi-coherent sheaves. This is because Remmert and Stein's result imply that the $Cone(X)$ of a projective analytic
subset of dimension $k$, $X\subset\cp^n$ is an analytic subset of dimension $k+1$ in $\co^{n+1}$, where the cone is defined by
\begin{equation}
Cone(X):=\pi^{-1}[X]\cup\lbrace0\rbrace,\hspace{0.3cm}\pi:\co^{n+1}\setminus\lbrace 0\rbrace\rightarrow\cp^n.
\end{equation}
\noindent Here $\pi$ is the usual projection of $\co^{n+1}$ on to the
projective space, so clearly $Cone(X)=\overline{\pi^{-1}[X]}$, and since $\pi$
is an analytic projection, $\pi^{-1}[X]$ is an analytic subset. Then from this
point on, the classical proof is to use the fact that the cone is homothetic-invariant to
show that the ideal of locally defined holomorphic functions that vanish at the
cone has a countable basis. Then with Hilbert's basis theorem it is easy to prove the
fact that the ring of germs of holomorphic functions is Noetherian. This shows
that $Cone(X)$ is in fact algebraic see \cite{Chirka}, but the same result can
be proved without algebraic methods with an equally simple proof using only
the geometric and analytical tools that we have presented thus far.  We start
the proof of Chow's theorem by citing another consequence of the proper mapping
theorem and we give a sketch of the proof.
\begin{theorem}\label{bishop}[Bishop]
        Let $X$ be a purely $k$ dimensional subvariety of $\con$ and
        let $B(R,0)$ be the standard ball in $\con$ of
        radius $R$. If there is  a constant $C\in\re^{+}$ such that
        \begin{equation}
                Vol_{2k}(X\cap B(R,0))\leq CR^{2k} \hspace*{0.2cm}\forall\, R\in\re^{+},
        \end{equation} then $X$ is algebraic.
\end{theorem}
\section{Foliations on Kähler manifolds and limits of leafs}

\noindent In this section I apply Bishop's sequence theorem (theorem \ref{bishop sequence}) to prove a version of \cite{EMS}[Theorem 1], for the particular case of compact complex foliations on K\"ahler manifolds with all leaves compact. Finally, we end with an example of such a foliation.

\begin{theorem}\label{kahlerEMS}
        Let $M$ be a compact connected Kähler manifold
        of complex dimension $n$ and $\mathfrak{F}$ a holomorphic foliation
        with leaves of complex dimension $d<n$ and with all leaves compact,
        then:
\begin{enumerate}
        \item[1] The $2d$-dimensional volume (with
                respect to the Kähler metric) of the leaves is uniformly
                bounded.
        \item[2] The quotient space $M/\mathfrak{F}$ is a complex orbifold, with singularities corresponding to leaves
                with non-trivial holonomy (which by the first proposition is a finite group).
\end{enumerate}
\end{theorem}
\begin{proof}
First, we prove the continuity of the volume function $\nu:M\rightarrow\re^{+}$ given by
$z\mapsto Vol_{2d}(\mathcal{L}_z)$, for the set of generic leaves:
$$
H_0=\lbrace x\in M\,\vert\,\mathcal{L}_x\text{ has zero holonomy}\rbrace,
$$
where $\mathcal{L}_x$ denotes the leaf through $X$. We note here that the set $H_0$ is a dense
open set of $M$ (see \cite{EMT}). Let $\lbrace z_n\rbrace\subset H_0$ be a
sequence such that each $z_i$ is on a different leaf $\mathcal{L}_{z_i}$ of
$\mathfrak{F}$ and such that $z_n\rightarrow z\in H_0$. Since all the leaves
are compact we have that $\mathcal{L}_{z_i}\rightarrow\mathcal{L}$ for the
Hausdorff metric, where $\mathcal{L}\subset M$ is a non-empty closed set.
Now, let $\mathcal{L}_z$ be the leaf containing $z$, since $\mathcal{L}_z$ has zero
holonomy, by the generalization of Reeb's stability theorem (see \cite{Thurston}), there exists a tubular neighborhood of
$\mathcal{L}_z$, say $U$, which is biholomorphic to $\mathcal{L}_z\times D$,
where $D\subset\co^{n-d}$ is an open disk (ball) and such that $U$ is a
saturated open subset of $M$ with every leaf of $U$ mapped biholomorphically to
the sets $\mathcal{L}_z\times\lbrace x \rbrace$. Therefore, every leaf in $U$ is
homologous to $\mathcal{L}_z$ and by Corollary \ref{wirtinger}, all leaves in
$U$ have the same volume. Since $z\in U$ and $z_n\rightarrow z$ there is a
large enough positive integer $N$ such that all leaves $\mathcal{L}_{z_k}$ have
the same volume for $k>N$. Therefore, by theorem \ref{bishop sequence},
$\mathcal{L}$ is an analytic subvariety of $U$ of complex dimension $d$ with
its volume equal to $\lim_{n\rightarrow\infty} Vol_{2d}(\mathcal{L}_{z_n})$. Since
tangency to $\mathfrak{F}$ is defined locally by the null space of $d$ holomorphic
1-forms, by Hausdorff convergence this tangency
is preserved on the limit, so $\mathcal{L}$ is tangent to $\mathfrak{F}$ and therefore $\mathcal{L}=\mathcal{L}_z$. Now, $\nu$
is not continuous in general but rather lower semicontinuous. Semicontinuity can
be proved by showing that the leaf space $M/\mathfrak{F}$ is Hausdorff (see \cite{EMS}[p. 20]),
which we will prove, but more than that, the volume
function $\nu$ is in fact discretely lower-semicontinuous, meaning that for any
$n\in\zah^+$, $z\in M$ and $\epsilon\in\re^+$, there is a small enough
neighborhood of $z$ such that
$$
\text{either }\nu(y)>n\,\nu(z)\hspace{0.2cm}\text{or}\hspace{0.2cm}\vert\nu(y)-k\,\nu(z)\vert<\epsilon\hspace{0.2cm}\text{for
some }\,k\in\lbrace 1,\cdots,n\rbrace.
$$
We prove this fact locally. Given a tubular neighborhood of $\mathcal{L}_z$,
$W$ there is a bundle retraction $\rho: W\rightarrow\mathcal{L}_z$ with $\rho^{-1}(x)$
homeomorphic to a disk. For every leaf $\mathcal{L}_y$, the restriction $\rho\vert_{W\cap\mathcal{L}_y}:(W\cap\mathcal{L}_y)\rightarrow\mathcal{L}_z$
is a codimension zero submersion, if $y$ is sufficiently close to $z$, then
$\mathcal{L}_y\subset W$ and also the image under $\rho$ of the leaf $\mathcal{L}_y$ covers all of $\mathcal{L}_z$.
Therefore, by compactness and analyticity, $\mathcal{L}_y$ is a finitely sheeted covering space of $\mathcal{L}_z$ with covering transformation
$\rho\vert_{\mathcal{L}_y}$ which proves the discrete lower-semicontinuity of $\nu$.
We note that the previous proof is also true for $X\subset M$ a compact saturated set.  With
this, we proceed to show that the set where $\nu$ is not bounded, also known as
the ``bad set":
$$
B:=\lbrace x\in M\,\vert\, \nu\,\text{is not bounded in a neighborhood of }x\rbrace,
$$
is a saturated compact set of codimension greater or equal to 2 (see \cite{Epstein}). For
each $x\in B$ there exists a chart around $x$, $U\subset M$ such that
$\nu\vert_U$ is bounded. Therefore, given a convergent sequence in $U$,
say $x_n\rightarrow x$, we have that their corresponding leaves in $\lbrace
U\cap\mathcal{L}_n\rbrace$ (which we can suppose to be generic) converge to
$U\cap B$ by the fact that $B$ is a compact saturated set. Therefore, since the
volume of $\lbrace U\cap\mathcal{L}_n\rbrace$ is uniformly bounded, by theorem
\ref{bishop sequence}, we have that $U \cap B$ is an analytic subvariety of
complex dimension $d$. This means that $B$ is an analytic set of $M$ with real
codimension at least 2, which means that $M\setminus B$ is a connected open
subset since $M$ is connected. Then, by discrete lower semicontiuity the volume is therefore bounded.
%Note that if we continued with this line of argumentation we find that Epstein's fibration defined as: $\lbrace B_{\alpha}\rbrace$  with $B_0=B$
 %, $B_{n+1}=\lbrace x\in B_{n}\,\vert\, \nu \text{ is unbounded in  }B_{n}\rbrace$ for finite ordinals
 %and $B_{\alpha}=\bigcap_{\beta<\alpha}B_{\beta}$ for infinte ordinals, then there is a finite ordinal $k\in\nat$
 %such that $B_k=\emptyset$.

%%By the local boundeness of $\nu$ in $M\setminus B$ and Ehresmann's structure theorem, the set $(M\setminus B)\setminus H_0$ is a countable union of smooth manifolds of codimension $\geq 2$. Let $\lbrace \Sigma_n\rbrace_{n\in\nat}$ be the components of $M\setminus B$, then for each $\sigma_k$ we can find a sequence $\rbrace\mathcal{L}_m\rbrace$ of generic leaves such that $\mathcal{L}_m\rightarrow\Sigma_k$

The second assertion follows easily from Thurston's genralization of Reeb's theorem \cite[Thurston], since for
every leaf with null holonomy $\mathcal{L}$ we have an open laminated set $U$
biholomorphic to $\mathcal{L}_z\times D$ so locally $M/\mathfrak{F}$ is homeomorphic to $D$.
Furthermore, $M/\mathfrak{F}$ is Hausdorff since every leaf is
compact, so if $\mathcal{L}_1$ and $\mathcal{L}_2$ are two distinct leaves, them
there are $\lbrace\epsilon_1,\epsilon_2\rbrace\subset\re^+$ such that the sets
$$
D_i:=\lbrace z\in M\,\vert\, d_H(\mathcal{L}_i,z)<\epsilon_i\rbrace,\hspace{0.2cm} i\in\lbrace1,2\rbrace,
$$
\noindent are disjoint, so intersecting  with a laminated tubular neighborhood
of $\mathcal{L}_{i}$ we have that $M/\mathfrak{F}$ is Hausdorff. Finally, if
$\mathcal{L}$ has non-trivial holonomy, then by the boundedness of the Volume function, the
holonomy group $H(\mathcal{L})$ is finite (see \cite{EMS} [p. 20]) and
$M/\mathfrak{F}$ is locally homeomorphic to $D/H(\mathcal{L})$, where
$\mathcal{L}$ has a tubular neighborhood  homeomorphic to $\mathcal{L}\times D$.
\end{proof}
\begin{thebibliography}{3}

\bibitem{A-F} Alarc\'on, A., Forstneric (2019) \textit{A foliation of the ball by complete holomorphic discs}, Mathematische Zeitschrift,
Springer-Verlag, pp. 169-174.

\bibitem{A-V} Alexander J.C., Verjovsky A., (1988) \textit{First Integrals for Singular Holomorphic Foliations With Leaves of Bounded Volume}
Holomorphic Dynamics. Lecture Notes in Mathematics, vol 1345. Springer, Berlin, Heidelberg, pp. 1–10.

\bibitem{Beuville} Beauville, A. (2000) \textit{Complex manifolds with split tangent bundle}, Complex analysis and algebraic geometry
, de Gruyer, Berlin, 61-70.

\bibitem{ZB} Blocki, Z. (2011) \textit{The Complex Monge-Amp\`ere Equation in K\"ahler Geometry} Lecture Notes in Mathematics 2075
part of Pluripotential Theory, Springer-Verlag, pp. 95-143.

\bibitem{Bishop} Bishop, E. (1964) \textit{Conditions for the Analyticity  of certain sets}, Michigan Math. J. 11, No. 4, 289--304.

\bibitem{Bondal} Bondal, A. I,. (1993) \textit{Non-commutative deformations and Poisson brackets on projetive spaces},
Max-Planck-Institut f\"ur Matematik, Germany, 93–67.

\bibitem{brunella} Brunella, M., (2015) \textit{Birational Geometry of Foliations}, IMPA Monographs,
Springer International Publishing, Switzerland.

\bibitem{Chirka} Chirka E. M. (1989) \textit{Complex Analytic Sets}, Kluwer
Academic, Dordrecht, The Netherlands.

\bibitem{Chow} Chow, W-L. (1949) \textit{On Compact Complex Analytic Varieties},
American Journal of Mathematics, Vol. 71, No. 4, pp. 893-914.

\bibitem{D-SS} Donaldson, S., Sun, S. (2014) \textit{Gromov-Hausdorff Limits of K\"ahler Manifolds and Algebraic Geometry I},
 Acta Math 213, Springer-Verlag, 63-106.

\bibitem{DPPT}Druel S., Pereira J V., Pym B., Touzet F. \textit{A global Weinstein splitting theorem for
holomorphic Poisson manifolds}, Por publicarse.

\bibitem{EMS} Edwards, R., Millet, K., Sullivan, D. (1975) \textit{Foliations
With All Leaves Compact}, Topology Vol. 16, Pergamon Press, 1977, pp. 13-32.

\bibitem{V-A} Epstein, D. B. A., Millet, K. C., Tischler D.
(1977) \textit{Leaves Without Holonomy}, Journal of The London Mathematical Society-second Series, 548-552.

\bibitem{E-V} Epstein, D. B. A., Voght, E. (1978) \textit{A Counterexample to the Periodic Orbit Conjecture},
Annals of Mathematics, Vol. 108, pp. 539-552.

\bibitem{Epstein1} Epstein, D. B. A. (1976) \textit{Foliations with all leaves compact, Annales de Institute Fourier},
tome 26 no. 1, pp. 265-282.

\bibitem{Epstein2} Epstein, D. B. A. (1972) \textit{Periodic Flows on Three-Manifolds}, Annals of Mathematics,
Second Series, Vol. 95, No. 1 (Jan., 1972), pp. 66-82.

\bibitem{Gua-Pym} Gualtieri, M., Pym, B. (2012) \textit{Poisson modules and degeneracy loci},
Proceedings of the London Mathematical Society 107(3), pp. 627-654.

\bibitem{R-S} Remmert R., Stein, K. (1953) \textit{Über die wesentlichen
Singulariäten analyscher Mengen}. Math. Annalen, Bd. 126, S. 263--306.

\bibitem{Sullivan} Sullivan, D. (1976) \textit{A counterexample to the  periodic orbit conjecture}, Publications mathématiques de I.H.É.S., tome 46, pp. 5-14.

\bibitem{Myers} Myers, S. B. (1941), \textit{Riemannian manifolds with positive mean curvature}, Duke Mathematical Journal,
8 (2): 401–404.

\bibitem{SCHARK} Schark, I. J.(1961), \textit{Maximal Ideals in an Algebra of Bounded Analytic Functions}, The Institute for Advanced Study
Princeton, New Jersey,pp. 735-746.

\bibitem{Stolzenberg} Stolzenberg G. (1966) \textit{Volumes, Limits and
Extensions of Analytic Varieties}, Lecture Notes in Mathematics,
Springer-Verlag, Berlin.

\bibitem{Thurston} Thurston W., P. (1974) \textit{A Generalization of Reeb Stability Theorem}, Topology Vol. 13,
Pergamon Press, Great Britain, pp. 347-352.

\bibitem{Pereira} Pereira J. V., (2001) \textit{Global Stability for Holomorphic Foliations in Kaehler Manifolds}.
Qualitative Theory of Dynamical Systems volume 2, pp. 381–384.

\bibitem{S-Yau} Yau, S-T., (1978) \textit{On The Ricci Curvature of a Compact K\"ahler Manifold and the Complex Monge-Amp\`ere Equation, I*},
Communications on Pure and Applied Mathenatics, Vol. XXXXI, John Wiley \& Sons. Inc, 339-411.
\end{thebibliography}

\end{document}
