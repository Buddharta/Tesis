% Created 2022-12-27 mar 00:56
% Intended LaTeX compiler: pdflatex
\documentclass[letterpaper]{book}
\usepackage[utf8]{inputenc}
\usepackage[T1]{fontenc}
\usepackage{graphicx}
\usepackage{grffile}
\usepackage{longtable}
\usepackage{wrapfig}
\usepackage{rotating}
\usepackage[normalem]{ulem}
\usepackage{amsmath}
\usepackage{textcomp}
\usepackage{amssymb}
\usepackage{capt-of}
\usepackage{hyperref}
\usepackage{graphicx}
\usepackage{amsmath, amsthm, amssymb}
\usepackage[table, xcdraw]{xcolor}
\usepackage{pst-node}
\usepackage{tikz-cd}
\usepackage{amsmath}
\usepackage{float}
\usepackage{amsfonts}
\usepackage[spanish, activeacute, english]{babel}
\usepackage{amscd}
\usepackage{color}
\usepackage{transparent}
\graphicspath{{./figs/}}
\usepackage{makeidx}
\usepackage{afterpage}
\usepackage{array}

\newtheorem{theorem}{Theorem}[section]
\newtheorem{prop}[theorem]{Proposition}
\newtheorem{cor}[theorem]{Corollary}
\newtheorem{lema}[theorem]{Lema}
\newtheorem{definition}{Definition}[section]
\newtheorem{conjeture}{Conjeture}

\renewcommand{\figurename}{Figure}
\newcommand{\zah}{\ensuremath{ \mathbb Z }}
\newcommand{\nat}{\ensuremath{ \mathbb N }}
%%\newcommand{\dem}{{\sc Demostraci\'on. }}
\newcommand{\bg}{\ensuremath{\overline \Gamma}}
\newcommand{\ga}{\ensuremath{\Gamma}}
\newcommand{\fb}{\ensuremath{\overline f}}
\newcommand{\la}{\ensuremath{\lambda}}
\newcommand{\La}{\ensuremath{\Lambda}}
\newcommand{\bt}{\ensuremath{\overline T}}
\newcommand{\li}{\ensuremath{\mathbb{L}}}
\newcommand{\ord}{\ensuremath{\mathbb{O}}}
\newcommand{\co}{\ensuremath{\mathbb C }}
\newcommand{\con}{\ensuremath{\mathbb{C}^n}}
\newcommand{\cp}{\ensuremath{\mathbb{CP}}}
\newcommand{\rp}{\ensuremath{\mathbb{RP}}}
\newcommand{\re}{\ensuremath{\mathbb R }}
\newcommand{\hc}{\ensuremath{\widehat{\mathbb C} }}
\newcommand{\pslz}{\ensuremath{PSL(2,\mathbb Z) }}
\newcommand{\pslr}{\ensuremath{PSL(2,\mathbb R) }}
\newcommand{\pslc}{\ensuremath{PSL(2,\mathbb C) }}
\newcommand{\hd}{\ensuremath{\mathbb H^2}}


\author{Carlos Eduardo Martinez Aguilar}
\title{Tesis: Holomorphic Foliations with bounded volumes in Kähler Manifolds}
\begin{document}

\maketitle
\tableofcontents
\newpage

\chapter{Preliminaries}
Some notation that will be used throughout the thesis: $\con$ will be the vector space of dimension $n$ over $\co$ the set of complex numbers. Given an open set $U\subset\con$ the set of holomorphic functions $f:U\rightarrow\co$ will be denoted as $\mathcal{H(U)}$ and $\mathcal{O}_{U}$ will represent the sheaf of holomorphic germs over $U$.
For a set of functions $\{f_{1},\dots,f_{k}\}\subset\mathcal{O}_{U}$ its vanishing set its defined as
\[
  V(f_{1},\dots,f_{k}):=\{x\in U\,\vert\,f_{1}=\dots =f_{k}=0\}.
\]
The set of complex lines in a $d$ dimensional complex vector spaces, also known as the complex projective space, will be denoted as $\cp^{d-1}$. Complex and real manifolds will be denoted usually with the letters $L, N, Ñ, M,$. etc. The tanget space of a manifold wil be denoted as $TM$
\section{Complex manifolds, analytic varieties and their limits}
\subsection{Kähler manifolds and varieties}

Kähler manifolds are special because of the relationship between their riemannian metrics and their associated symplectic $(1,1)$-forms, one aspect of this relationship is the fact that both both asociated volume forms coincide.
\begin{definition}
        Let $M$ be a complex manifold with with respect to the integrable complex structure
        $J:TM\rightarrow TM$. A \textit{Hermitian metric} on $M$ is a smooth family
        real bilinear forms $\lbrace h_p\rbrace_p$ for all $p\in M$ where
        $h_p:T_pM\times T_pM\rightarrow\co$, such that
\begin{itemize}
        \item $h_p(JX,JY)=h_p(X,Y)\hspace{0.3cm}\forall\lbrace X,Y\rbrace\subset T_pM\hspace{0.3cm}\forall p\in M.$
        \item $h_p(X,JX)>0\hspace{0.3cm}\forall X\in T_p M\setminus\lbrace 0\rbrace\hspace{0.3cm}\forall p\in M.$
        \item $h_p(X,Y)=\overline{h_p(Y,X)}\hspace{0.3cm}\forall\lbrace X,Y\rbrace\subset T_pM\hspace{0.3cm}\forall p\in M.$
\end{itemize}
We will also denote
$h(\cdot,\cdot)=\langle\cdot,\cdot\rangle$. We call a manifold with a Hermitian
metric a \textit{Hermitian manifold}.
\end{definition}
We note here that a Hermitian metric $h$ has an associated Riemannian metric and a associated 2-form given by the real and
imaginary part of $h$ and vice versa. A Riemannian manifold of even dimension $(M,g)$ with a complex structure $J$ has
a natural Hermitian metric given by $$h(X,Y):=g(X,Y)-i\omega(X,Y).$$
Where $\omega$ is its associated 2-form given by $\omega(X,Y)=g(JX,Y)$ which is clearly antisymmetric since $J^2=-Id_M$.
\begin{definition}
        Let $(M,h)$ be a Hermitian manifold with Hermitian metric
        $h$ and associated 2-form $\omega$, $h$ is a Kähler metric if
        $$d\,\omega=0.$$ We call a manifold with a Kähler metric a \textit{Kähler manifold}.
        The associated 2-form will be called a \textit{Kähler form} or a
        \textit{Kähler symplectic form}.
\end{definition}
Being a Kähler manifold has some strong topological restrictions. For example, powers of the Kähler
form $\omega^k$ are non trivial representatives of cohomology classes in
$H^{2k}(M ;\re)$, meaning these groups are never trivial. More important
for our purposes is the fact that a complex submanifold of a Kähler
manifold is also K\"ahler.
\begin{definition}
        We say $N$ is a complex embedded submanifold of $M$ if $N$ is a complex manifold such that there exist $f:N\rightarrow M$
        a holomorphic immersion.
\end{definition}
        By our definition of submanifold, it follows that every complex submanifold of a Kähler manifold is also Kähler
        since the clossedness of the 2-form follows from the compatibility of $J$ and the derivative under the pullback
        $f^*$. Other consequences are the following:

        \begin{theorem}[Wirtinger's inequality]
        Let $M$ be a Kähler manifold with Kähler form $\omega_M$ and let $f:N\rightarrow M$ be a closed oriented immersion
        of an oriented real manifold of real dimension $2k$. Let $\omega=f^*\,\omega_m$, then
        \begin{equation}
                        \frac{\omega^k}{k!}\leq dVol_N\hspace{0.3cm}\text{where }dVol_N\text{ is the volume form of N,}
        \end{equation}
        and the equality holds if and only if $N$ is a complex submanifold of $M$.
\end{theorem}
Related to this is the following results due to Wirtinger that, among other things, implies that Kähler submanifols of a Kähler manifolds minimize the volumes of their respective homological class.

\begin{theorem}
        Any complex submanifold of a Kähler manifold is a minimal submanifold.
\end{theorem}
\begin{cor}\label{wirtinger}
        Let $N$ be a complex compact submanifold with boundary of a Kähler manifold $M$, then $N$ is a volume minimizing
        submanifold in its homology class $H_{2k}(M,\partial N,\zah),$ meaning that for any real submanifold $X$ of
        real dimension $2k$ and boundary $\partial N$ homologous to $N$ has $$Vol_{2k}(N)\leq Vol_{2k}(X).$$
\end{cor}

Volume forms: Let $\omega=-Im\langle\cdot,\cdot\rangle$ be the standard 2-form of the standard euclidean Kähler metric in $\con$ and let $M$ be a complex submanifold of $\con$ by Wirtinger's inequality, if $Vol_{2k}(M)$ is the volume of $M$ given by the riemannian structure $g=Re\langle\cdot,\cdot\rangle\vert_{M}$, then
\begin{equation}
        Vol_{2k}(M)= \frac{1}{k!}\int_M \omega^k.
\end{equation}
If $X$ is a purely $k$ dimensional analytic subset of $\con$ and $\Sigma(X)$ is
its singular locus, then $M=X\setminus\Sigma(X)$ is a complex manifold and
since $\Sigma(X)$ is an analytic subset of $X$ of lesser dimension its volume is negible, i.e.
\[
    Vol_{2k}(X)= \frac{1}{k!}\int_{x} \omega^k=
                              \frac{1}{k!}\int_{X\setminus\Sigma(X)} \omega^k=
                                                            Vol_{2k}(X\setminus\Sigma(X)).
\]
An \emph{analytic variety} is defined as a paracompact Hausdorff space $X$ with a locally ringed structure $\mathcal{H}_{X}$ modeled by the local ring structure of
a vanishing set in some finite dimensional complex vector space, that is, $(X,\mathcal{H}_{X})$ locally isomorphic to a local model $(Z,\mathcal{O}_{U}/(f_{1},\dots,f_{k}))$, where $U\subset\con$ is an open set and $Z=V(f_{1},\dots,f_{k})$, and $(f_{1},\dots,f_{k})$ denotes the ideal generated by the functions $\{f_{1},\dots,f_{k}\}$. Given a open set $V\subset X$ and an open set $U\subset\co$ and an isomorphism between ringed structures as discussed, $\varphi:U\rightarrow V$, sometimes the pair $(V,\varphi)$ will be called a \emph{chart}.

Moreover it is said that an analytic variety $X$ is of pure dimension $k$ if all of its irreducible components are of the same dimension. Similarly the definition of a complex algebraic variety is the same, only considering the ring of polynomials $\co[z_{1},\dots,z_{n}]$ as its structure ring.

An \emph{analytic subspace} of an analytic variety $(X,\mathcal{H}_{X})$ consist of a subset $Y\subset X$ such that for every $y\in Y$ there is a chart $(U,\varphi)$ around $y$ with
\[
  \varphi(Y\cap U)=V_U(\varphi(\eta_1),\dots,\varphi(\eta_k)),
\]

\noindent and $\lbrace\eta_1,\dots,\eta_k\rbrace\subset\mathcal{H}_{X}$, naturally we have a local ring structure $(Y,\mathcal{H}_{Y})$ given by $\mathcal{H}_Y:=\mathcal{H}_n/\mathcal{I}(Y)$, as before.
Note that using projective coordinates, the set of vanishing points of homogeneous polynomials are well defined on $\cp^n$. Analytic and algebraic subsets of $\cp^n$ are going to be called  \emph{\textbf{projective} analytic subsets} and \emph{\textbf{projective} algebraic subsets} respectively.
\subsection{Volumes of Analytic varieties and Hausdorff limits}
\end{document}


\begin{thebibliography}{3}

\bibitem{A-F} Alarc\'on, A., Forstneric (2019) \textit{A foliation of the ball by complete holomorphic discs}, Mathematische Zeitschrift,
Springer-Verlag, pp. 169-174.

\bibitem{A-V} Alexander J.C., Verjovsky A., (1988) \textit{First Integrals for Singular Holomorphic Foliations With Leaves of Bounded Volume}
Holomorphic Dynamics. Lecture Notes in Mathematics, vol 1345. Springer, Berlin, Heidelberg, pp. 1–10.

\bibitem{Beuville} Beauville, A. (2000) \textit{Complex manifolds with split tangent bundle}, Complex analysis and algebraic geometry
, de Gruyer, Berlin, 61-70.

\bibitem{ZB} Blocki, Z. (2011) \textit{The Complex Monge-Amp\`ere Equation in K\"ahler Geometry} Lecture Notes in Mathematics 2075
part of Pluripotential Theory, Springer-Verlag, pp. 95-143.

\bibitem{Bishop} Bishop, E. (1964) \textit{Conditions for the Analyticity  of certain sets}, Michigan Math. J. 11, No. 4, 289--304.

\bibitem{Bondal} Bondal, A. I,. (1993) \textit{Non-commutative deformations and Poisson brackets on projetive spaces},
Max-Planck-Institut f\"ur Matematik, Germany, 93–67.

\bibitem{brunella} Brunella, M., (2015) \textit{Birational Geometry of Foliations}, IMPA Monographs,
Springer International Publishing, Switzerland.

\bibitem{Chirka} Chirka E. M. (1989) \textit{Complex Analytic Sets}, Kluwer
Academic, Dordrecht, The Netherlands.

\bibitem{Chow} Chow, W-L. (1949) \textit{On Compact Complex Analytic Varieties},
American Journal of Mathematics, Vol. 71, No. 4, pp. 893-914.

\bibitem{D-SS} Donaldson, S., Sun, S. (2014) \textit{Gromov-Hausdorff Limits of K\"ahler Manifolds and Algebraic Geometry I},
 Acta Math 213, Springer-Verlag, 63-106.

\bibitem{DPPT}Druel S., Pereira J V., Pym B., Touzet F. \textit{A global Weinstein splitting theorem for
holomorphic Poisson manifolds}, Por publicarse.

\bibitem{EMS} Edwards, R., Millet, K., Sullivan, D. (1975) \textit{Foliations
With All Leaves Compact}, Topology Vol. 16, Pergamon Press, 1977, pp. 13-32.

\bibitem{V-A} Epstein, D. B. A., Millet, K. C., Tischler D.
(1977) \textit{Leaves Without Holonomy}, Journal of The London Mathematical Society-second Series, 548-552.

\bibitem{E-V} Epstein, D. B. A., Voght, E. (1978) \textit{A Counterexample to the Periodic Orbit Conjecture},
Annals of Mathematics, Vol. 108, pp. 539-552.

\bibitem{Epstein1} Epstein, D. B. A. (1976) \textit{Foliations with all leaves compact, Annales de Institute Fourier},
tome 26 no. 1, pp. 265-282.

\bibitem{Epstein2} Epstein, D. B. A. (1972) \textit{Periodic Flows on Three-Manifolds}, Annals of Mathematics,
Second Series, Vol. 95, No. 1 (Jan., 1972), pp. 66-82.

\bibitem{Gua-Pym} Gualtieri, M., Pym, B. (2012) \textit{Poisson modules and degeneracy loci},
Proceedings of the London Mathematical Society 107(3), pp. 627-654.

\bibitem{R-S} Remmert R., Stein, K. (1953) \textit{Über die wesentlichen
Singulariäten analyscher Mengen}. Math. Annalen, Bd. 126, S. 263--306.

\bibitem{Sullivan} Sullivan, D. (1976) \textit{A counterexample to the  periodic orbit conjecture}, Publications mathématiques de I.H.É.S., tome 46, pp. 5-14.

\bibitem{Myers} Myers, S. B. (1941), \textit{Riemannian manifolds with positive mean curvature}, Duke Mathematical Journal,
8 (2): 401–404.

\bibitem{SCHARK} Schark, I. J.(1961), \textit{Maximal Ideals in an Algebra of Bounded Analytic Functions}, The Institute for Advanced Study
Princeton, New Jersey,pp. 735-746.

\bibitem{Stolzenberg} Stolzenberg G. (1966) \textit{Volumes, Limits and
Extensions of Analytic Varieties}, Lecture Notes in Mathematics,
Springer-Verlag, Berlin.

\bibitem{Thurston} Thurston W., P. (1974) \textit{A Generalization of Reeb Stability Theorem}, Topology Vol. 13,
Pergamon Press, Great Britain, pp. 347-352.

\bibitem{Pereira} Pereira J. V., (2001) \textit{Global Stability for Holomorphic Foliations in Kaehler Manifolds}.
Qualitative Theory of Dynamical Systems volume 2, pp. 381–384.

\bibitem{S-Yau} Yau, S-T., (1978) \textit{On The Ricci Curvature of a Compact K\"ahler Manifold and the Complex Monge-Amp\`ere Equation, I*},
Communications on Pure and Applied Mathenatics, Vol. XXXXI, John Wiley \& Sons. Inc, 339-411.
\end{thebibliography}
