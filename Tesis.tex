\documentclass[12pt,twoside,a4paper]{report}

\usepackage{graphicx} % Required for inserting image
\usepackage{subcaption} % Required for subfigures
\usepackage[T1]{fontenc} %%% for bold small cap

\graphicspath{{./figs/}}
\usepackage[margin=2.5cm]{geometry}  % to adjust margin 
\usepackage{hyperref}   % for hyper references
% \addbibresource{Reference.bib}

%%%%%%% Header-Footer %%%%%%%%%%%%%%%
\usepackage{fancyhdr}
\renewcommand{\chaptermark}[1]{\markboth{#1}{}}
\renewcommand{\sectionmark}[1]{\markright{#1}}
%%%%%%%%%%%%%%%%%%%%%%%%%%%%%%%%%%%%%%%

\usepackage[activeacute, english]{babel}
\usepackage{amsmath, amsthm, amssymb, amsfonts}
\usepackage[table, xcdraw]{xcolor}
\usepackage[utf8]{inputenc}
\usepackage[normalem]{ulem}
\usepackage{grffile}
\usepackage{longtable}
\usepackage{wrapfig}
\usepackage{rotating}
\usepackage{textcomp}
\usepackage{capt-of}
\usepackage{hyperref}
\usepackage{graphicx}
\usepackage{pst-node}
\usepackage{tikz-cd}
\usepackage{float}
\usepackage{amscd}
\usepackage{color}
\usepackage{transparent}
\usepackage{makeidx}
\usepackage{afterpage}
\usepackage{array}

\newtheorem{definition}{Definition}[section]
\newtheorem{theorem}{Theorem}[section]
\newtheorem{prop}[theorem]{Proposition}
\newtheorem{cor}[theorem]{Corollary}
\newtheorem{lema}[theorem]{Lema}
\newtheorem{conjeture}{Conjeture}
\newtheorem{example}{Example}

\renewcommand{\figurename}{Figure}
\newcommand{\zah}{\ensuremath{ \mathbb Z }}
\newcommand{\nat}{\ensuremath{ \mathbb N }}
\newcommand{\bg}{\ensuremath{\overline \Gamma}}
\newcommand{\ga}{\ensuremath{\Gamma}}
\newcommand{\fb}{\ensuremath{\overline f}}
\newcommand{\la}{\ensuremath{\lambda}}
\newcommand{\La}{\ensuremath{\Lambda}}
\newcommand{\bt}{\ensuremath{\overline T}}
\newcommand{\li}{\ensuremath{\mathbb{L}}}
\newcommand{\ord}{\ensuremath{\mathbb{O}}}
\newcommand{\co}{\ensuremath{\mathbb C }}
\newcommand{\con}{\ensuremath{\mathbb{C}^n}}
\newcommand{\cp}{\ensuremath{\mathbb{CP}}}
\newcommand{\rp}{\ensuremath{\mathbb{RP}}}
\newcommand{\re}{\ensuremath{\mathbb R }}
\newcommand{\hc}{\ensuremath{\widehat{\mathbb C} }}
\newcommand{\pslz}{\ensuremath{PSL(2,\mathbb Z) }}
\newcommand{\pslr}{\ensuremath{PSL(2,\mathbb R) }}
\newcommand{\pslc}{\ensuremath{PSL(2,\mathbb C) }}
\newcommand{\hd}{\ensuremath{\mathbb H^2}}

%%%%%%%%% Title,Author,Date    %%%%%%%%%%%%%%%%%%%%%%%%%%%%%%%%%%%%%%%%%%
\author{Carlos Eduardo Martinez Aguilar}
\title{Complex-Analytic aspects of Holomorphic Foliations with bounded volumes in Kähler Manifolds}
\date{July 2024}
%%%%%%%%%%%%%%%%%%%%% Personal details as New commands %%%%%%%%%%%%%%%%%%%%%%%%%%%%%%%%%%%%
\newcommand{\thesistitle}{Complex-Analytic aspects of Holomorphic Foliations with bounded volumes in Kähler Manifolds}
\newcommand{\myname}{Carlos Eduardo Martínez Aguilar}
\newcommand{\thesisdate}{July 2024}
%%%%%%%%%%%%%%%%%%%%%%%%%%%%%%%%%%%%%%%%%%%%%%%%%%%%%%%%%%%%%%%%%%%%%%%%%
\begin{document}
\pagenumbering{gobble} % Turns off page numbering
\begin{titlepage}
    \centering
     {\huge \textbf{\thesistitle} \par}
      \vspace{2cm}
     {\large \sc A Thesis \par}
      \vspace{0.2cm}
     {\large \sc Submitted For The Degree Of \par}
     \vspace{0.2cm}
      {\large \sc {\textbf{Doctor of Philosophy}} \par} 
     \vspace{0.2cm}
     {\large \sc Instituto de Matemáticas UNAM  \par}
     \vspace{1cm}
     {\large  by \par}
     \vspace{0.5cm}
     {\large \textbf{\myname} \par}
     \vspace{0.5cm}
    \includegraphics[width=2.5cm,height=2.5cm]{~/Templates/figs/unam.jpg}\\ % Logo
    \vspace{2cm}
    {\large \sc Department of Mathematics \par}
     {\large \sc IMATE UNAM  \par}
     {\large \sc Mexico City, Mexico \par}
    \vspace{0.5cm}
    {\large \sc \thesisdate \par}
    \vspace{2cm}
\end{titlepage}

%%%%%%%% Blank Page  %%%%%%%%%%%%%%%%
\clearpage
\thispagestyle{empty}
%%%%%%%%%%%%%%%%%%%%%%%%%%%%%%%%%%%%%

%%%%%%%%%%%%%%%%%%%%%%%% Dedication page %%%%%%%%%%%%%%%%%%%%%%%%%%%%%%%%
\clearpage
\thispagestyle{empty}
\begin{center}
    \vspace*{\fill}
    This thesis is dedicated to\\
    My parents, Midori and Alberto Verjovsky.
    \vspace*{\fill}
\end{center}
\clearpage
%%%%%%%%%%%%%%%%%%%%%%%%%%%%%%%%%%%%%%%%%%%%%%%%%%%%%%%%%%%%%%%%%%%%%%%%%


%%%%%%%%%%%%%%%%%%% Acknowledgemnents Page %%%%%%%%%%%%%%%%%%%%%%%%%%%%%%%
\clearpage
\thispagestyle{plain}
\pagenumbering{roman} % Turns off page numbering
\begin{center} 
    \huge\textbf{Acknowledgements}\\
\end{center}
    \normalsize
    I would like to acknowledge everyone in my lab.
\clearpage
%%%%%%%%%%%%%%%%%%%%%%%%%%%%%%%%%%%%%%%%%%%%%%%%%%%%%%%%%%%%%%%%%%%%%%%%%
%%%%%%%%%%%% Contents, Abbreviations, ListofFigures %%%%%%%%%%%%%%%%%%%%%


\tableofcontents %%%%% Table of contents %%%%%%%%
\listoffigures  %%%%% List of figures %%%%%%%%
\listoftables   %%%%%% List of Tables %%%%%%%%%
%%%%%%%%%%%%%%%%%%%%%%%%%%%%%%%%%%%%%%%%%%%%%%%%%%%%%%%%%%%%%%%%%%%%%%%%%


%%%%%%%%%%%%%%%%%%%%%%%%   New Chapter   %%%%%%%%%%%%%%%%%%%%%%%%%%%%%%%%
\chapter{Introduction}
\pagenumbering{arabic}
\pagestyle{fancy}
\fancyhf{}
\fancyhead[LE]{\rightmark}
\fancyhead[RO]{\leftmark}
\fancyfoot[C]{\thepage}
\noindent The Purpose of this work is to show how the study of analytic aspects of the leaves of holomorphic foliations shows the analogies between the foliations and holomorphic maps. But more importantly, this thesis main purpose is to give its author the title of doctor in mathematics.

This work is the result of the research conducted during the course of my doctoral studies, both as a doctoral student and PhD candidate later, the main idea behind that I've worked on and developed durring this period is; analytic varieties, foliations and holomorphic functions are the same type of objects but of different kind. By being the same type of object I mean that they share a core of proposition that are true for all of them if they are enunciated properly. In this sense one can think of holomorphic foliations as the geometric version of the classical theory of holomophic/meromorphic functions. 

Together with my thesis advisor Dr.Santiago Alberto Verjovsky Solá, we studied holomorphic foliations $\mathfrak{F}$ whose leaves satisfy certain geometric properties that ensure a certain regularity in the structure of the leaf space of the foliation. We refer to regularity in the sense that normally the leaf space is topologically complex (non-Hausdorff), making it necessary to restrict the type of dynamics on the leaves of the foliation to guarantee some topological regularity in the leaf space. An example of these properties is the case of foliations whose leaves have bounded volumes in K\"ahler manifolds $(M,h)$.

The historical basis for studying foliations with bounded volume comes from previous works by mathematicians such as Epstein, Edwards, Millet, Reeb, Sullivan, Haefliger, J.C. Alexander, and Dr. Verjovsky himself, among others; see \cite{EMS}, \cite{V-A}, \cite{E-V}, \cite{Epstein1}, \cite{Epstein2}. However, the research we have proposed so far is supported by the previous work of J.C. Alexander and Dr. Verjovsky \cite{A-V}, which in turn is based on works by Errett Bishop \cite{Bishop} on extensions and Hausdorff limits of sequences of analytic varieties and follows the spirit of the presentation of Bishop's results found in the book "Volumes, Limits, and Extensions of Analytic Varieties" by Gabriel Stolzenberg \cite{Stolzenberg}.

More precisely, we have been investigating ways to extend the range of possible applications of these results to various areas of complex analysis and complex geometry, with an emphasis on the theory of holomorphic foliations. We have found connections between known theorems of complex analysis and complex geometry. One of the first findings of these connections is, for example, a novel proof of Chow's theorem \cite{Chow}.

First, the basic propperties of analytic sets and holomorphic manifolds are stablished, then there is a discussion on their analytic propperties, particularly about integration, volume and dimension of this spaces. This naturarly leads to the definitions of the so called ``\emph{Lelong numbers}'' and of ``\emph{currents}'' as integration operators on analytic sets, which serve as analogues of residues and differential forms.

Then, these propperties are used in the contexts of holomphic foliations, to study foliations with all leaves compact in Kähler manifolds, subsequently the spaces of Douady and Bartlet are introduced as a sort of universal leaf space. These special spaces serve the purposse of stablishing the analytic structure of the leaf space of a holomorphic foliation in a compact holomorphic manifolds for which the leaf space is hausdorff beyond just a metric stucture given by the Gromov-Hausdorff metric.

\chapter{Complex Analytic Varieties and Manifolds}
\pagenumbering{arabic}
\pagestyle{fancy}
\fancyhf{}
\fancyhead[LE]{\rightmark}
\fancyhead[RO]{\leftmark}
\fancyfoot[C]{\thepage}
This chapter is for stablishing the basic definitions, notations and conventions that will be used throughout the thesis. 

The complex vector space of simension $n$ will be denoted as $\con$ and the set of complex numbers naturally will be denoted as $\co$. Given an open set $U\subset\con$ the set of holomorphic functions $f:U\rightarrow\co$ will be denoted as $\mathcal{H}(U)$ and $\mathcal{O}_{U}$ will represent the sheaf of holomorphic germs over $U$. Similarly, the set of \emph{biholomorphic} funtions between two spaces $U$ and $V$ i.e. the set of biyective holomorphic mappings with holomorphic inverse between $U$ and $V$ as $\mathrm{Bil}(U,V)$, if $U=V$, then it will be denoted as $\mathrm{Bil}(U).$

The set of complex lines in a $d$ dimensional complex vector spaces, also known as the complex projective space, will be denoted as $\cp^{d-1}$. Complex and real manifolds will be denoted usually with the letters $L, N, Ñ, M,\dots$. etc. The tanget space of a manifold wil be denoted as $TM$ and the space of smooth sections (vector fields) of $TM$ will be denoted as $\mathcal{T}M$.

An (almost)-complex structure of an even dimenstional real manifold will be denoted as $J:TM\rightarrow TM$, with $J²=-id_{TM}$. Moreover the complexification of a even dimentional real manifold will be denoted as $TM_{\co}:=TM\otimes\co$, and the complex conjugate splitting will be denoted as $TM_{\co}=TM^{1,0}\oplus TM^{0,1}$.
Similarly the space of $(p,q)$-forms on a complex manifold $M$ is denoted as $\Omega^{(p,q)}(M)$, defined as the section space of the vector bundles
\[
  \bigwedge^{p,q}T^{*}M=\bigwedge^{p}T^{*}M\otimes_{\co}\bigwedge^{q}T^{*}M,
\]
\noindent where the following \emph{``conjugation''} convention is asumed: $\overline{\Omega^{(p,q)}(M)}=\Omega^{(q,p)}(M)$, with the space of complex differential forms given as the space
\[
  \Omega^{k}_{\co}(M)=\bigoplus_{p+q=k}\Omega^{p,q}(M).
\]
\noindent The external differential operator is $d=\partial + \overline{\partial}$, where $\partial:\Omega^{(p,q)}(M)\rightarrow \Omega^{(p+1,q)}(M)$ is the holomorphic differential operator and $\overline{\partial}:\Omega^{(p,q)}(M)\rightarrow\Omega^{(p,q+1)}(M)$ is the anti-holomorphic differential operator.\\
Unless stated otherwise a submanifold will be understood as an embedded submanifold, i.e. $N$ is an embedded (differential/complex) submanifold of $M$ if $N$ is a (differential/complex) manifold such that there exist $f:N\rightarrow M$ a (differental/holomorphic) immersion.
A foliation $\mathfrak{F}$ on a manifold $M$ of real dimension $n$, can be defined in a numerous of distict and interesting ways, depending on the application needed a foliation will be understood as
\begin{itemize}
  \item A decomposition of $M$ into disjoint connected submanifolds called \emph{leaves},
        \[
          M=\bigcup_{\alpha}\mathcal{L_{\alpha}},
        \]
        \noindent such that for every point $p\in M$ there exists an open set and a ($\mathcal{C}^{k}$, differential, analytic or holomorphic) function $(U,\phi)$, with $\phi\rightarrow\re^{m}$, such that every $\mathcal{L}_{\alpha}\cap U$ is a level set of $\phi$.
  \item A \emph{foliated chart} $\{(U_{i},\varphi_{i})\}$ such that $\{U_{i}\}$ is a covering of $M$ and $\varphi_{i}:U_{i}\rightarrow\re^{n}$ is a ($\mathcal{C}^{k}$, differential, analytic or holomorphic) chart such that
        \[
          \varphi_{i}(U_{i})=W_{i}\times\re^{m},\text{ with }W_{i}\subset\re^{n-m}.
        \]
      \item An \emph{involutive} distribution of the tangent bundle $TM$, i.e. A $\mathcal{C}^{\infty}(M)$-module $\mathfrak{F}\subset\mathcal{T}M$ closed under the lie bracket
        \[
          [X,Y]\in\mathfrak{F}\text{ for any } \{X,Y\}\subset\mathfrak{F}
        \]
  \item The dual poin of view from the previous viewpoint will also be adopted, i.e. a foliation $\mathfrak{F}$ is a set of $k$-forms that nulls on an involutive disribution.
\end{itemize}
If all the leaves are of the same dimension, then the foliation is said to be \emph{regular}, note that a foliated chart determines the dimension of the leaves. This is equivalent to all the spaces generated by the poinwise evaluations of $\mathcal{F}$ being of the same dimension.\\

The \emph{holonomy grupoid} also called the \emph{graph} of a foliation $(M,\mathcal{F})$ is the set of all curves contained in the leaves of the foliation \textit{modulo homotopy} and will be denoted as 
\[
G(M,\mathfrak{F}):=\{[\gamma],\,\gamma:[0,1]\rightarrow\mathcal{L}\,\,\vert\,\,\mathcal{L}\in\mathfrak{F}\}.
\]
Since grupoids are a difficult structure to work with, there is also the easier to work with \emph{holonomy group}, defined at a point $z\in L$ as
\[
\mathcal{H}_z(L):=\{[\gamma]\in G(M,\mathfrak{F})\,|\,\gamma(0)=\gamma(1)=z\}.
\]
This is naturally a group and can be representen via a surjective homomorphism of the fondamental group of $L$, said surjective homomorphism is called the \emph{holonomy morphism}
\[
H:\pi_{1}(L,z)\rightarrow\mathcal{H}_z(L).
\]
The prefix on the holonomy groups will be preserved in order to not confuse it with the set of holomorphic functions on a set.
\section{Complex manifolds and analytic varieties}
For a set of functions $A\subset\mathcal{O}_{U}$, where $U\subset M$ is an open set on a complex manifold, the vanishing set of is denoted and defined as
\[
  V(A):=\{x\in U\,\vert\,f(x)=0,\,\forall\,f\in A\}.
\]
An \emph{analytic variety} is defined as a paracompact Hausdorff space $X$ with a locally ringed structure $\mathcal{H}_{X}$ modeled by the local ring structure of a vanishing set in some finite dimensional complex vector space, that is, $(X,\mathcal{H}_{X})$ locally isomorphic to a local model $(Z,\mathcal{O}_{U}/(f_{1},\dots,f_{k}))$, where $U\subset\con$ is an open set and $Z=V(f_{1},\dots,f_{k})$, and $(f_{1},\dots,f_{k})$ denotes the ideal generated by the functions $\{f_{1},\dots,f_{k}\}$. Given a open set $V\subset X$ and an open set $U\subset\co$ and an isomorphism between ringed structures as discussed, $\varphi:U\rightarrow V$, sometimes the pair $(V,\varphi)$ will be called a \emph{chart}.

Moreover it is said that an analytic variety $X$ is of pure dimension $k$ if all of its irreducible components are of the same dimension. Similarly the definition of a complex algebraic variety is the same, only considering the ring of polynomials $\co[z_{1},\dots,z_{n}]$ as its structure ring.

An \emph{analytic subspace} of an analytic variety $(X,\mathcal{H}_{X})$ consist of a subset $Y\subset X$ such that for every $y\in Y$ there is a chart $(U,\varphi)$ around $y$ with
\[
  \varphi(Y\cap U)=V_U(\varphi(\eta_1),\dots,\varphi(\eta_k)),
\]

\noindent and $\lbrace\eta_1,\dots,\eta_k\rbrace\subset\mathcal{H}_{X}$, naturally we have a local ring structure $(Y,\mathcal{H}_{Y})$ given by $\mathcal{H}_Y:=\mathcal{H}_n/\mathcal{I}(Y)$, as before.
Note that using projective coordinates, the set of vanishing points of homogeneous polynomials are well defined on $\cp^n$. Analytic and algebraic subsets of $\cp^n$ are going to be called  \emph{\textbf{projective} analytic subsets} and \emph{\textbf{projective} algebraic subsets} respectively.

\subsection{Kähler manifolds}
A K\"ahler manifold is a complex manifold $(M,I)$ with an additional Hermitian structure $h=g-i\omega$ such that the (1,1)-form defined by $\omega=-\Im (h)$ is closed $d\,\omega=0$. This closed form is denoted as the \emph{K\"ahler form}, i.e., $M$ is a real differentiable manifold of dimension $2d$ and $I$ is a complex structure $I^2=-1$, furthermore $h$ is a Hermitian product on the tangent space of $M$.

It is worth mentioning that $g=\Re (h)$ is a Riemannian metric with the same volume form as the one induced by $\omega$, this is due to the natural relation between $\omega$ and $g$ given by $g(\cdot,\cdot)=\omega(\cdot,I(\cdot))$. The most important prototypical examples of K\"ahler manifolds are manifolds that I have already mentioned, these are $\con$ with the traditional Hermitian inner product and $\cp^{n}$ with the metric known as the Fubini-Study metric.

Kähler manifolds are special because of the relationship between their riemannian metrics and their associated symplectic $(1,1)$-forms, one aspect of this relationship is the fact that both both asociated volume forms coincide.
\begin{definition}
        Let $M$ be a complex manifold with with respect to the integrable complex structure
        $J:TM\rightarrow TM$. A \textit{Hermitian metric} on $M$ is a smooth family
        real bilinear forms $\lbrace h_p\rbrace_p$ for all $p\in M$ where
        $h_p:T_pM\times T_pM\rightarrow\co$, such that
\begin{itemize}
        \item $h_p(JX,JY)=h_p(X,Y)\hspace{0.3cm}\forall\lbrace X,Y\rbrace\subset T_pM\hspace{0.3cm}\forall p\in M.$
        \item $h_p(X,JX)>0\hspace{0.3cm}\forall X\in T_p M\setminus\lbrace 0\rbrace\hspace{0.3cm}\forall p\in M.$
        \item $h_p(X,Y)=\overline{h_p(Y,X)}\hspace{0.3cm}\forall\lbrace X,Y\rbrace\subset T_pM\hspace{0.3cm}\forall p\in M.$
\end{itemize}
An Hermitean metric $h$ will also be denoted as $\langle\cdot,\cdot\rangle$, a manifold with a Hermitian metric $(M,h)$ its called a \textit{Hermitian manifold}.
\end{definition}
An Hermitian metric $h$ has an associated Riemannian metric and a associated $(1,1)$-form given by the real and
imaginary part of $h$ respectively. Conversely, a Riemannian manifold of even dimension $(M,g)$ with a complex structure $J$ has
a natural Hermitian metric given by
\[
  h(X,Y):=g(X,Y)-i\omega(X,Y).
\]
\noindent Where $\omega$ is its associated 2-form given by $\omega(X,Y)=g(JX,Y)$ which is clearly antisymmetric since $J^2=-id_M$.
\begin{definition}
        Let $(M,h)$ be a Hermitian manifold with Hermitian metric $h$ and associated $(1,1)$-form $\omega$, $h$ is a Kähler metric if
        \[
          d\omega=0,
        \]
        \noindent an Hermitean manifold with a Kähler metric is called a \textit{Kähler manifold}, and its associated $(1,1)$-form is  called a \textit{Kähler form} or a \textit{Kähler symplectic form}.
\end{definition}
\begin{example}
  A classic example that will be used al the time in this thesis is the complex projetive space of dimesion $n$ $\cp^{n}$ with the Fubini-Study metric, which in homogeneous coordinates $Z=[1:\zeta_{1}\dots:\zeta_{n}]=[1:\zeta]$ is defined by the Kähler form
  \[
    \omega_{FS}=\frac{i}{2}\partial \overline{\partial}\log(|Z|^{2})=\frac{i}{2}\partial \overline{\partial}\log(1+|\zeta|^{2}).
  \]
\end{example}

It is common to definitionne a K\"ahler metric via its K\"ahler form, in the case of the Fubini-Study metric, it is defined in coordinates $Z=[1:\zeta_{1},\dots,\zeta_{n}]=[1:\zeta]$ as (see \cite[p. 154]{Chirka})
\[
    \omega=\frac{i}{2}\partial\overline{\partial}\log(|Z|^{2})=\frac{i}{2}\partial\overline{\partial}\log(1+|\zeta|^{2}).
\]

Likewise, it is important to mention that every complex submanifold of a K\"ahler manifold is K\"ahler. Similarly, compact complex submanifolds of a K\"ahler manifold minimize the volume in their homology class. This is some evidence that Kähler manifolds have some strong topological restrictions, another way to express this is the fact that powers of the Kähler form $\omega^k$ are non trivial representatives of cohomology classes in $H^{2k}(M ;\re)$, meaning these groups are never trivial. More importantly complex submanifolds of a Kähler manifold are also K\"ahler.
Another important consequence is the following famous result
\begin{theorem}[Wirtinger's inequality]
        Let $M$ be a Kähler manifold with Kähler form $\omega_M$ and let $f:N\rightarrow M$ be a closed oriented immersion
        of an oriented real manifold of real dimension $2k$. Let $\omega=f^*\,\omega_M$, then
        \begin{equation}
                        \frac{\omega^k}{k!}\leq d\textrm{Vol}_N\hspace{0.3cm}\text{where }d\textrm{Vol}_N\text{ is the volume form of N,}
        \end{equation}
        and the equality holds if and only if $N$ is a complex submanifold of $M$.
\end{theorem}
\subsection{Kähler manifolds and their volumes}
\noindent Another way to state Wirtinger's inequality is the following
\begin{theorem}\label{wirtinger}
        Let $N$ be a complex compact submanifold with boundary of a Kähler manifold $M$, then $N$ is a volume minimizing
        submanifold in its homology class $H_{2k}(M,\partial N,\zah),$ meaning that for any real submanifold $X$ of
        real dimension $2k$ and boundary $\partial N$ homologous to $N$ has
        \[
          \textrm{Vol}_{2k}(N)\leq \textrm{Vol}_{2k}(X).
        \]
\end{theorem}
\noindent Let $\omega=-Im\langle\cdot,\cdot\rangle$ be the standard $(1,1)$-form of the standard euclidean Kähler metric in $\con$ and let $M$ be a complex submanifold of $\con$ by Wirtinger's inequality, if $\textrm{Vol}_{2k}(M)$ is the volume of $M$ given by the riemannian structure $g=Re\langle\cdot,\cdot\rangle\vert_{M}$, then
\begin{equation}
        \textrm{Vol}_{2k}(M)= \frac{1}{k!}\int_M \omega^k.
\end{equation}
If $X$ is a purely $k$ dimensional analytic subset of $\con$ and $\Sigma(X)$ is
its singular locus, then $M=X\setminus\Sigma(X)$ is a complex manifold and
since $\Sigma(X)$ is an analytic subset of $X$ of lesser dimension its volume is negible, i.e.
\[
    \textrm{Vol}_{2k}(X)= \frac{1}{k!}\int_{x} \omega^k=
                              \frac{1}{k!}\int_{X\setminus\Sigma(X)} \omega^k=
                                                            \textrm{Vol}_{2k}(X\setminus\Sigma(X)).
\]
\subsection{The Hausdorff measure and The Hausdorff metric}
\noindent Besides the volumes of manifolds and analytic sets, a very useful tool that is directly related to the volumes is the Hausdorff measure, which can be conceptualized as a generalization of the volume for arbitrary dimensions and is in itself a tool for measuring dimensions as an arbitrary non-negative real number.
The $\delta$-Hausdorff measure of a subset $S\subset\con$ is denoted by $\mathcal{H}_{\delta}(S)$, here is a list some of the most important properties of this measure, see \cite{Stolzenberg}[ch. 3]

\begin{enumerate}
        \item[1.] If $\mathcal{H}_{\delta}(S)<\infty$, and $\delta<\kappa$, then $\mathcal{H}_{\kappa}(S)=0$.\\

        \item[2.] If $f:X\rightarrow Y$ is a Lipschitz continuous function with
        Lipschitz constant $\lambda$, then for any $\delta\in\re^{+}$ and
        $S\subset X$, the following inequality holds
        \[
        \mathcal{H}_{\delta}(f(S))\leq\lambda^{\delta}\mathcal{H}_{\delta}(S).
        \]
        \item[3.]If $X=\re^{n}$ and $S$ is a smooth submanifold of dimension
        $k\in\zah^{+}$, then the volume of $S$ as a submanifold is related to
        its Hausdorff measure by the formula
        \[
        \textrm{Vol}_{k}(M)=\alpha_{k}\mathcal{H}_k(S),\hspace*{0.3cm}\alpha_k= \frac{\pi^k}{\Gamma(k/2+1)},
        \]
\noindent where $\Gamma(z)$ is the Euler's gamma function.
\end{enumerate}

Related to the Hausdorff measure is the Hausdorff metric defined for $K_1$
and $K_2$ compact subsets of a metric space $(X,d)$ as $$
d_h(K_1,K_2):= \max_{x\in K_1}\left\{d(x,K_2)\right\}+\max_{y\in K_2}\left\{d(y,K_1)\right\}.$$

The Hausdorff metric allows to definitionne a convenient notion of convergence of closed subsets: let $\{S_n \}$ be anysequence of closed subsets of $X$, then $S_n$ converges to the closed subset $S$ in the Hausdorff metric $S_n\overset{h}\rightarrow S$, if for every $K\subset X$ compact we have
\[
  d_h(K\cap S_n,K\cap S)\rightarrow 0.
\]
\section{Bishop's theorems}
\subsection{Analytic and Algebraic Varieties}
\begin{theorem}[Chow]\label{Chow}
        Every closed analytic subspace of complex projective space $\cp^{n}$ is algebraic.
\end{theorem}
\noindent We demonstrated this based on the little-known results of Bishop about analytic varieties of pure dimension $k$. Here, an \emph{analytic space} or \emph{analytic variety} is a Hausdorff paracompact topological space $X$ with a ringed structure $\mathcal{H}_X$, i.e., a sheaf of rings on $X$ that is locally isomorphic to an analytic subset of $\con$. More precisely, a set $X$ is analytic if for each $x\in X$ there is an open neighborhood $V$ and a homeomorphism $\varphi_V:V\rightarrow Z,$ where $Z\subset U$, $U$ is an open subset of $\con$ and $Z=Z(f_1,\dots,f_k)(U)$ represents the analytic subset of the zeros of the holomorphic functions $f_i:U\rightarrow\co$. Furthermore, it must be that the pullback $\varphi^{*}:\mathcal{H}(U)/\mathcal{I}(Z)\rightarrow \mathcal{H}_X(V)$
is an isomorphism of rings between $\mathcal{H}_X(V)$ and the quotient ring of
\[
        \mathcal{H}(U)=\{f:U\rightarrow\co\,|\,f \text{ is holomorphic}\}\quad\text{with}\quad\mathcal{I}(Z)=\{f\in\mathcal{H}(U)\,|\,f|_{Z}=0\}.
\]
Thus, an analytic subspace $Y$ of $X$ is a ringed space $(Y,\mathcal{H}_Y)$ with a closed embedding $\iota:Y\rightarrow X$ such that locally the ring
$\mathcal{H}_Y$ is isomorphic to the quotient ring of $\mathcal{H}$ with the ideal of functions that are zero on the image of $Y$, where the isomorphism is given by the
pullback $\iota^{*}$. Similarly, the definition of an algebraic set is obtained by replacing the ring of analytic functions $\mathcal{H}(U)$ with the ring of
polynomials in $\con$ denoted by $\co[z_1,\dots,z_n]$.
\subsection{Volume Growth a messure for algebraicity}
\noindent Following Stolzenberg's presentation (see \cite{Stolzenberg}), the result linking
Bishop's results with Chow's theorem is a result that states the following:
\begin{theorem}[Bishop]\label{Bishop1}
        Let $X\subset\con$ be an analytic subvariety of pure complex dimension $k$. If for every $R\in\re^+$
        $$\textrm{Vol}_{2k}(X\cap B_R(0))\leq CR^{2k},$$
        where $B_R(0)$ denotes the ball centered at the origin with radius $R$, $C\in\re^{+}$ is a positive constant
        and $\textrm{Vol}_{2k}$ is the $2k$-dimensional volume, then $X$ is algebraic.
\end{theorem}
\noindent The proof we gave in our article of Chow's theorem is of a very different character from Serre's famous proof (see \cite{GAGA}) and speaks of an interesting connection between analysis and algebraic geometry.

\subsection{Bishop's Sequential Theorem}
\noindent The connections between complex analysis and complex geometry have guided us to explore further the possible connections between these two branches,
as well as other results with complex analysis. One example is the following classical result that we proved
based on Bishop's results.
\begin{theorem}[Montel]\label{Montel}
        Let $B\subset\con$ be the unit ball, if $\mathcal{H}(\overline{B})$ is the Banach algebra of holomorphic functions that are continuous
        on the boundary of $B$, then every locally bounded family of functions $F\subset\mathcal{H}(\overline{B})$ is a \textit{normal} family.
\end{theorem}
As a reminder, a family of functions $F$ is \textit{normal} if and only if its closure is sequentially compact.
If we think of Montel's theorem from the perspective of Bishop's theorems, we can consider that the following Bishop's theorem is a
generalization (see \cite[p. 30]{Stolzenberg}).

\begin{theorem}[Bishop]\label{Bishop2}
        Let $\{ V_n \}_{n\in\nat}$ be a sequence of analytic subvarieties of pure complex dimension $k$ in a region
        $\Omega\subset\con$ with volume uniformly bounded by a constant $C\in\re^{+}$.
        If the sequence $V_n\rightarrow V$ converges to a closed set $V\subset\con$ in the Hausdorff sense, then $V$ is an
        analytic variety.
\end{theorem}
This is because Dr. Verjovsky and I proved Montel's result using theorem \ref{Bishop2} by considering the graphs of the
functions as analytic varieties
\[
        \Gamma_{f}=\{(z,w)\in B\times\co\,|\,w=f(z)\}=Z(w-f(z)),
\]
\noindent where $B$ is the unit ball in $\con$ and $Z(g)=\{z\in B\,|\,g(z)=0\}$ denotes the zero set of a holomorphic function $g:B\rightarrow\co$. Here, we think of $\Gamma_{f}$ as pure-dimensional analytic varieties in an open subset of $\co^{n+1}$, and we proved
that for every sequence $\{f_n\}_{n\in\nat}\subset F$ of functions in a locally bounded family, there is a subsequence
whose graphs converge in the sense of the Hausdorff metric. Additionally, we proved that its limit set
is in turn the graph of a holomorphic function in $B$.

The convergence of a sequence of sets $S_n\rightarrow S$ \emph{in the Hausdorff sense} for closed sets of a metric space $(X,d)$
occurs when $S_n\cap K\rightarrow S\cap K$ in the Hausdorff metric $d_H$ for every compact $K\subset X$, where it is defined as
\[
        d_H(K_1,K_2):= \max_{x\in K_1}\left\{d(x,K_2)\right\}+\max_{y\in K_2}\left\{d(y,K_1)\right\}.
\]

\subsection{Similarities between the theories of holomorphic functions and complex analytic sets}

\noindent From the perspective that I have shown so far, it can be appreciated that Bishop's results can be thought of as generalizations of classical results from complex analysis, but in the context of the theory of analytic varieties.

\begin{table}[hpt]
        \caption{Table with similarities between classical theorems of complex variable and Bishop's theorems.}\label{Tab}
        \centering
        \begin{tabular}{|m{5.5cm}|m{5.5cm}|} \hline
                        \begin{center} \vspace*{0.2cm}
                                \underline{\textbf{Complex Analysis}}
                        \end{center} &
                        \begin{center} \vspace*{0.2cm}
                                \underline{\textbf{Analytic sets in $\con$}}
                        \end{center} \\
                \hline
                \begin{center}
                        \textit{Liouville's Theorem.}
                \end{center} &
                \begin{center}
                        \textit{Bishop's Theorem (Theorem \ref{Bishop1}).}
                \end{center}\\
                        \hline If $\vert f(z)\vert\leq C\,R^k$ in the set $\{\vert z\vert\leq R\}$
                        for all $R\in\re^{+}$ with $f$ entire and $k\in\zah^{+}$, then
                        $f$ is a polynomial.
                        &
                        \vspace{0.1cm}
                        If $\textrm{Vol}_{2k}(X\cap B(R,0))\leq CR^{2k}$ for all
                        $R\in\re^{+}$, where $X$ is an analytic subvariety in $\con$, then $X$ is algebraic.\\
                        \hline
                        \vspace{0.1cm}
                        \begin{center}
                        \textit{Riemann's Extension Theorem.}
                \end{center}
                        &
                \begin{center}
                        \textit{Bishop's generalization of the Remmert-Stein theorem (see \cite{R-S} and \cite[p. 34]{Stolzenberg}).}
                \end{center} \\
                        \hline If $f:(\Omega\setminus E)\subset\co\rightarrow\co$ is a holomorphic function and $E$ is a compact subset
                        of capacity $0$, then $f$ can be extended to a holomorphic function
                        in the entire region $\Omega$.
                        &
                        \vspace{0.1cm}
                        Let $U\subset\con$ be a bounded open subset of $\con$ and let $B\subset U$ be a closed subset
                        such that $X\subset U\setminus B$ is a pure dimensional subvariety of dimension $k$ with $B\subset\overline{X}$.
                        If $B$ has capacity $0$ relative to the algebra of analytic functions on $X$ that are
                        continuous on $\overline{X}$ and if there exists a proper map $f:U\rightarrow\co^k$ on $B$ such that $f(B)$
                        is not an open connected subset of $\co^k$, then $\overline{X}\cap U$ is an analytic subset
                        of $U$ (see \cite[Theorem 4]{Bishop}).\\
                        \hline
                \begin{center}
                        \textit{Montel's Compactness Theorem.}
                \end{center}
                        &
                \begin{center}
                        \textit{Bishop's theorem on sequences of analytic varieties with uniformly bounded volume.}
                \end{center}\\
                \hline
                \vspace{0.1cm}
                Let $\lbrace\Gamma_i\rbrace$ be a sequence of graphs of uniformly bounded holomorphic
                functions, $f_i:\Delta\rightarrow\co$ such that $\Gamma_i\overset{d_H}\longrightarrow\Gamma$ (Hausdorff convergence),
                where $\Gamma\subset\co^2$ is a closed subset and $\Delta$ is the unit disk in $\co$,
                then $\Gamma$ is the graph of a holomorphic function.
                        &
                Let $\lbrace V_i\rbrace$ be a sequence of analytic subvarieties of $\Omega\subset\con$ with uniformly
                bounded volumes such that $V_i\overset{d_H}\longrightarrow V\subset\Omega$ in the Hausdorff sense, then
                $V$ is an analytic subvariety of $\con$ (\cite[p. 30]{Stolzenberg}). \\ \hline
        \end{tabular}
\end{table}

Moreover, analytic varieties themselves can clearly be thought of as a generalization of holomorphic functions,
since each holomorphic function $f:\Omega\subset\con\rightarrow\co$ has associated to it the analytic variety given by its zero divisor $[Z]=Z(g)$
or as previously mentioned, by its graph $\Gamma_{f}\subset\Omega\times\co$, the first being an analytic subvariety of
dimension $n-1$ and the second being
an analytic subvariety of pure dimension $n$. In addition to this, it is widely known that in the classical context
of the complex variable, every holomorphic function on a compact holomorphic variety is constant, for example $\cp^n$ is compact,
and therefore every global holomorphic function is constant. However, as we know, $\cp^n$ has many analytic subvarieties, as many
as sets of algebraically independent homogeneous polynomial functions. Table \ref{Tab} lists a series of
versions of classical theorems from complex analysis and their counterparts in the context of analytic varieties.




\subsection{Analytic varieties and their Hausdorff limits}
The first of Bishop's results that is very useful for understanding some of the analytical properties of purely $k$-dimensional subvarieties is the following convergence theorem
\begin{theorem}[Sequence theorem] \label{bishop sequence}
  Let $\lbrace
        V_n\rbrace$ be a sequence of purely $k$-dimensional subvarieties of a
        domain $\Omega\subset\con$ such that $V_n\overset{h}\rightarrow V$, with
        $V\subset\Omega$ a closed subset. If $$ \textrm{Vol}_{2k}(V_n)\leq M
        \hspace*{0.2cm}\forall n\in\nat, $$
        \noindent for $M\in\re$ a positive constant, then for the Hausdorff
        measure we have $\mathcal{H}_{2k+1}(V)=0$, moreover $V$ is a purely
        $k$-dimensional analytic subvariety of $\Omega$.
\end{theorem}
As a direct application of this result, one can show the following very useful proposition (see \cite{Stolzenberg}[ch. 4]).
\begin{theorem}\label{bishop mapping}[Bishop's proper mapping theorem] Let
        $\Omega\subset\con$ be a domain that contains $0$ and let
        $S\subset\Omega$ be a closed subset. If $\mathcal{H}_{2k+1}(S)=0$, then there
        is a suitable coordinate change of $\con$, $(z_1,\ldots,z_n)$ and
        neighborhoods $\Omega_k\subset\co^k$ and $\Omega_{n-k}\subset\co^{n-k}$, such that
        $0\in\Omega_k\times\Omega_{n-k}\subset\Omega$ and the projection
        \[
          \pi_k:S\cap(\Omega_k\times\Omega_{n-k})\rightarrow\Omega_k\, ,\hspace{0.3cm}\pi_k(z,w):=z,
        \]
        \noindent is a proper map.
\end{theorem}
\noindent This applied to purely $k$-dimensional analytic
subvarieties implies the regular coordinates theorem, meaning the proper
mapping is a finite sheeted analytic covering.

\chapter{Applications of Bishop's theorems}

\noindent As mentioned, this result by Bishop can be used to prove many other important
results (see \cite{Stolzenberg}) one of the most significant is the proof of
Remmert-Stein's theorem, this was generalized and proved by Bishop in
\cite{Bishop}.

\begin{theorem}\label{Rem-Stein}[Remmert-Stein]
        Let $\Omega\subset\con$ be an open subset and $Y$ an analytic subset
        of $\Omega$ and let $X$ be a analytic subset of
        $\Omega\setminus Y$. If $Y$ is of dimension at most $k-1$ and $X$ is of pure dimension $k$,
        then the closure of $X$, $\overline{X}\cap\Omega$ is an analytic subset of $\Omega$.
\end{theorem}
This is an essential step towards the proof of Chow's theorem if one is trying to avoid using categorical methods and
quasi-coherent sheaves. This is because Remmert and Stein's result imply that the $\textrm{Cone}(X)$ of a projective analytic
subset of dimension $k$, $X\subset\cp^n$ is an analytic subset of dimension $k+1$ in $\co^{n+1}$, where the cone is defined by
\begin{equation}
\textrm{Cone}(X):=\pi^{-1}[X]\cup\lbrace0\rbrace,\hspace{0.3cm}\pi:\co^{n+1}\setminus\lbrace 0\rbrace\rightarrow\cp^n.
\end{equation}
\noindent Here $\pi$ is the usual projection of $\co^{n+1}$ on to the
projective space, so clearly $\textrm{Cone}(X)=\overline{\pi^{-1}[X]}$, and since $\pi$
is an analytic projection, $\pi^{-1}[X]$ is an analytic subset. Then from this
point on, the classical proof is to use the fact that the cone is homothetic-invariant to
show that the ideal of locally defined holomorphic functions that vanish at the
cone has a countable basis. Then with Hilbert's basis theorem it is easy to prove the
fact that the ring of germs of holomorphic functions is Noetherian. This shows
that $\textrm{Cone}(X)$ is in fact algebraic see \cite{Chirka}, but the same result can
be proved without algebraic methods with an equally simple proof using only
the geometric and analytical tools that we have presented thus far.  We start
the proof of Chow's theorem by citing another consequence of the proper mapping
theorem and we give a sketch of the proof.
\begin{theorem}\label{bishop}[Bishop]
        Let $X$ be a purely $k$ dimensional subvariety of $\con$ and
        let $B(R,0)$ be the standard ball in $\con$ of
        radius $R$. If there is  a constant $C\in\re^{+}$ such that
        \begin{equation}
                \textrm{Vol}_{2k}(X\cap B(R,0))\leq CR^{2k} \hspace*{0.2cm}\forall\, R\in\re^{+},
        \end{equation} then $X$ is algebraic.
\end{theorem}
\section{Foliations on Kähler manifolds}
\subsection{Compact leaves and finite volume}
\noindent In this section I apply Bishop's results to study foliated compact Kähler manifolds with compact leaves.
\begin{theorem}\label{kahlerEMS}
        Let $M$ be a compact connected Kähler manifold
        of complex dimension $n$ and $\mathfrak{F}$ a holomorphic foliation
        with leaves of complex dimension $d<n$ and with all leaves compact,
        then:
\begin{enumerate}
        \item[1] The $2d$-dimensional volume (with
                respect to the Kähler metric) of the leaves is uniformly
                bounded.
        \item[2] The quotient space $M/\mathfrak{F}$ is a complex orbifold, with singularities corresponding to leaves
                with non-trivial holonomy (which by the first proposition is a finite group).
\end{enumerate}
\end{theorem}
\begin{proof}
First, we prove the continuity of the volume function $\nu:M\rightarrow\re^{+}$ given by
$z\mapsto \textrm{Vol}_{2d}(\mathcal{L}_z)$, for the set of generic leaves:
$$
H_0=\lbrace x\in M\,\vert\,\mathcal{L}_x\text{ has zero holonomy}\rbrace,
$$
where $\mathcal{L}_x$ denotes the leaf through $X$. We note here that the set $H_0$ is a dense
open set of $M$ (see \cite{EMT}). Let $\lbrace z_n\rbrace\subset H_0$ be a
sequence such that each $z_i$ is on a different leaf $\mathcal{L}_{z_i}$ of
$\mathfrak{F}$ and such that $z_n\rightarrow z\in H_0$. Since all the leaves
are compact we have that $\mathcal{L}_{z_i}\rightarrow\mathcal{L}$ for the
Hausdorff metric, where $\mathcal{L}\subset M$ is a non-empty closed set.
Now, let $\mathcal{L}_z$ be the leaf containing $z$, since $\mathcal{L}_z$ has zero
holonomy, by the generalization of Reeb's stability theorem (see \cite{Thurston}), there exists a tubular neighborhood of
$\mathcal{L}_z$, say $U$, which is biholomorphic to $\mathcal{L}_z\times D$,
where $D\subset\co^{n-d}$ is an open disk (ball) and such that $U$ is a
saturated open subset of $M$ with every leaf of $U$ mapped biholomorphically to
the sets $\mathcal{L}_z\times\lbrace x \rbrace$. Therefore, every leaf in $U$ is
homologous to $\mathcal{L}_z$ and by Corollary \ref{wirtinger}, all leaves in
$U$ have the same volume. Since $z\in U$ and $z_n\rightarrow z$ there is a
large enough positive integer $N$ such that all leaves $\mathcal{L}_{z_k}$ have
the same volume for $k>N$. Therefore, by theorem \ref{bishop sequence},
$\mathcal{L}$ is an analytic subvariety of $U$ of complex dimension $d$ with
its volume equal to $\lim_{n\rightarrow\infty} \textrm{Vol}_{2d}(\mathcal{L}_{z_n})$. Since
tangency to $\mathfrak{F}$ is defined locally by the null space of $d$ holomorphic
1-forms, by Hausdorff convergence this tangency
is preserved on the limit, so $\mathcal{L}$ is tangent to $\mathfrak{F}$ and therefore $\mathcal{L}=\mathcal{L}_z$. Now, $\nu$
is not continuous in general but rather lower semicontinuous. Semicontinuity can
be proved by showing that the leaf space $M/\mathfrak{F}$ is Hausdorff (see \cite{EMS}[p. 20]),
which we will prove, but more than that, the volume
function $\nu$ is in fact discretely lower-semicontinuous, meaning that for any
$n\in\zah^+$, $z\in M$ and $\epsilon\in\re^+$, there is a small enough
neighborhood of $z$ such that
$$
\text{either }\nu(y)>n\,\nu(z)\hspace{0.2cm}\text{or}\hspace{0.2cm}\vert\nu(y)-k\,\nu(z)\vert<\epsilon\hspace{0.2cm}\text{for
some }\,k\in\lbrace 1,\cdots,n\rbrace.
$$
We prove this fact locally. Given a tubular neighborhood of $\mathcal{L}_z$,
$W$ there is a bundle retraction $\rho: W\rightarrow\mathcal{L}_z$ with $\rho^{-1}(x)$
homeomorphic to a disk. For every leaf $\mathcal{L}_y$, the restriction $\rho\vert_{W\cap\mathcal{L}_y}:(W\cap\mathcal{L}_y)\rightarrow\mathcal{L}_z$
is a codimension zero submersion, if $y$ is sufficiently close to $z$, then
$\mathcal{L}_y\subset W$ and also the image under $\rho$ of the leaf $\mathcal{L}_y$ covers all of $\mathcal{L}_z$.
Therefore, by compactness and analyticity, $\mathcal{L}_y$ is a finitely sheeted covering space of $\mathcal{L}_z$ with covering transformation
$\rho\vert_{\mathcal{L}_y}$ which proves the discrete lower-semicontinuity of $\nu$.
We note that the previous proof is also true for $X\subset M$ a compact saturated set.  With
this, we proceed to show that the set where $\nu$ is not bounded, also known as
the ``bad set":
$$
B:=\lbrace x\in M\,\vert\, \nu\,\text{is not bounded in a neighborhood of }x\rbrace,
$$
is a saturated compact set of codimension greater or equal to 2 (see \cite{Epstein}). For
each $x\in B$ there exists a chart around $x$, $U\subset M$ such that
$\nu\vert_U$ is bounded. Therefore, given a convergent sequence in $U$,
say $x_n\rightarrow x$, we have that their corresponding leaves in $\lbrace
U\cap\mathcal{L}_n\rbrace$ (which we can suppose to be generic) converge to
$U\cap B$ by the fact that $B$ is a compact saturated set. Therefore, since the
volume of $\lbrace U\cap\mathcal{L}_n\rbrace$ is uniformly bounded, by theorem
\ref{bishop sequence}, we have that $U \cap B$ is an analytic subvariety of
complex dimension $d$. This means that $B$ is an analytic set of $M$ with real
codimension at least 2, which means that $M\setminus B$ is a connected open
subset since $M$ is connected. Then, by discrete lower semicontiuity the volume is therefore bounded.
%Note that if we continued with this line of argumentation we find that Epstein's fibration defined as: $\lbrace B_{\alpha}\rbrace$  with $B_0=B$
 %, $B_{n+1}=\lbrace x\in B_{n}\,\vert\, \nu \text{ is unbounded in  }B_{n}\rbrace$ for finite ordinals
 %and $B_{\alpha}=\bigcap_{\beta<\alpha}B_{\beta}$ for infinte ordinals, then there is a finite ordinal $k\in\nat$
 %such that $B_k=\emptyset$.

%%By the local boundeness of $\nu$ in $M\setminus B$ and Ehresmann's structure theorem, the set $(M\setminus B)\setminus H_0$ is a countable union of smooth manifolds of codimension $\geq 2$. Let $\lbrace \Sigma_n\rbrace_{n\in\nat}$ be the components of $M\setminus B$, then for each $\sigma_k$ we can find a sequence $\rbrace\mathcal{L}_m\rbrace$ of generic leaves such that $\mathcal{L}_m\rightarrow\Sigma_k$

The second assertion follows easily from Thurston's genralization of Reeb's theorem \cite[Thurston], since for
every leaf with null holonomy $\mathcal{L}$ we have an open laminated set $U$
biholomorphic to $\mathcal{L}_z\times D$ so locally $M/\mathfrak{F}$ is homeomorphic to $D$.
Furthermore, $M/\mathfrak{F}$ is Hausdorff since every leaf is
compact, so if $\mathcal{L}_1$ and $\mathcal{L}_2$ are two distinct leaves, them
there are $\lbrace\epsilon_1,\epsilon_2\rbrace\subset\re^+$ such that the sets
$$
D_i:=\lbrace z\in M\,\vert\, d_H(\mathcal{L}_i,z)<\epsilon_i\rbrace,\hspace{0.2cm} i\in\lbrace1,2\rbrace,
$$
\noindent are disjoint, so intersecting  with a laminated tubular neighborhood
of $\mathcal{L}_{i}$ we have that $M/\mathfrak{F}$ is Hausdorff. Finally, if
$\mathcal{L}$ has non-trivial holonomy, then by the boundedness of the Volume function, the
holonomy group $H(\mathcal{L})$ is finite (see \cite{EMS} [p. 20]) and
$M/\mathfrak{F}$ is locally homeomorphic to $D/H(\mathcal{L})$, where
$\mathcal{L}$ has a tubular neighborhood  homeomorphic to $\mathcal{L}\times D$.
\end{proof}

Now, in compact Kähler manifolds, the statents ``there is at least \emph{one} compact leaf with finite homology'' is equivalent to ``\emph{all} leaves are compact and have finite holonomy''. Formally is the following result

\begin{theorem}\label{compact-leaf}
Let $M$ be a compact connected K\"ahler manifold of dimension $n$ and $\mathfrak{F}$ be a holomorphic foliation of codimension $q<n$ for which there is at least one compact leaf with finite holonomy, then all leaves are compact with finite holonomy an therefore their volumes are uniformly bounded.
\end{theorem}
\begin{proof}
  Let us denote by $\mathcal L$ to the compact leaf of $\mathfrak F$, by generalized Reeb stability there exist a saturated neighborhood $U$ of $\mathcal{L}$ such that $U$ is biholomorphic to a product $\mathcal{L}\times D$ with $D\subset\co^{q}$ an open connected set, and such that every leaf in $U$ is a finitely sheeted covering of $\mathcal{L}$. Now, let us definitionne the following set
\[
    \mathcal{U}:=\{x\in M\,\vert\, \text{ the leaf through $x$ is compact with finite holonomy}\},
\]
clearly by the previous $\mathcal{U}$ is not just a non-empty set but it is also an non-empty open subset of $M$. We show that this set is also closed to finish the proof. Let $\{z_{n}\}\subset U$ be a convergent sequence with limit $z_{n}\rightarrow z_{0}$.
Moreover let $\mathcal{L}_{n}$ be the leaf of the foliation through $z_{n}$ and $\mathcal{L}$ the leaf through $z_{0}$.
To show $z_{0}\in\mathcal{U}$ is enough to show $\mathcal{L}$ is compact, but this is clear since for any foliated chart $V$ around $z_{0}$ there is a subsecuence $(z_{k},\mathcal{L}_{k})$ in the intersection $V\cap\mathcal{U}$.
By hypothesis all leaves $\mathcal{L}_{k}$ are compact with finite holonomy group and therefore have all bounded volume, and by the same arguments previously discussed in the proof of \ref{kahlerEMS} it is clear that $V\cap\mathcal{L}_{k}\rightarrow V\cap\mathcal{L}$.
Therefore by Bishop´s theorem \ref{bishop sequence} $\{\mathcal{L}_{k}\}$ converges to a closed analytic subset that is tangent to the foliation, since $M$ is compact, this means that $\mathcal{L}_{k}\rightarrow\mathcal{L}$ and therefore $\mathcal{L}$ is a closed leaf with finite volume and therfore is compact with finite holonomy group. Therefore $z_{0}\in\mathcal{U}$ and therefore $\mathcal{U}=M$  since it is both open and closed.
\end{proof}


{\huge\noindent\textbf{Here I need to work in fitting for these adecuately:}}

\section{Limits of compact leaves of foliations in K\"ahler Manifolds}
\noindent In addition to complex geometry and complex analysis, another application of Bishop's theorems and in particular Theorem
\ref{Bishop2}, is the following result on holomorphic foliations in K\"ahler varieties with compact leaves and uniformly bounded volume.

\subsection{Stability in K\"ahler Manifolds and Beauville's Theorem}

\noindent Thus, the description of the following theorem is clear, which Dr. Verjovsky and I proved as part of the previously mentioned paper:
\begin{theorem}[Edwards, Millet and Sullivan]\label{EMS}
        Let $M$ be a connected compact K\"ahler manifold of complex dimension $n$, i.e. real dimension $2n$, and $\mathfrak{F}$ a holomorphic foliation by compact
        leaves of real dimension $2d$ where $d<n$, then:
        \begin{enumerate}
                \item[a)] The volume with respect to the K\"ahler metric of the leaves is uniformly bounded.
                \item[b)] The quotient space $M/\mathfrak{F}$ is a complex orbifold with singularities at the leaves of non-trivial holonomy.
        \end{enumerate}
\end{theorem}
In addition to this, we proved that the volume function $\nu:M/\mathfrak{F}\rightarrow\re^{+}$ defined by the volume
\[
        \nu(\mathcal{L}_z):=\textrm{Vol}_{2d}(\mathcal{L}_z)
\]
is discretely lower semi-continuous, i.e., for every $n\in\zah^{+}$ and $z\in M$ there exists a neighborhood $W$ of $z$
such that for every $\epsilon\in\re^{+}$
\[
        \nu(y)>n\nu(z)\hspace{0.2cm}\text{or}\hspace{0.2cm}|\nu(y)-k\nu(z)|<\epsilon\hspace{0.1cm}\forall y\in W\text{ and for some}\hspace{0.1cm}k\in\{1,\dots,n\}.
\]
\noindent Moreover, the jumps in continuity correspond to the leaves with non-trivial holonomy, which are covered
by leaves with trivial holonomy. Since all leaves are compact, the holonomy is finite and the volume of the leaves
with non-trivial holonomy is a fraction of the volume of the leaves with trivial holonomy by Reeb stability \cite{Thurston}.
Then, the generalized Reeb stability theorem \cite{Thurston} tells us how to obtain the coordinate charts of $M/\mathfrak{F}$.
In the case of leaves with trivial holonomy, for each leaf $\mathcal{L}$ there exists a saturated neighborhood, i.e., $U=\bigcup_{z\in U}\mathcal{L}_z$ of the leaf,
such that $U$ is biholomorphic to $\mathcal{L}\times B$ where $B\subset\co^{n-d}$ is a ball. Moreover, each leaf in $U$ is biholomorphic to $\mathcal{L}\times\{w\}$ with $w\in B$.
This means that $M/\mathcal{F}$ has a holomorphic coordinate chart to a ball of complex dimension $n-d$ around each leaf
with trivial holonomy. Now, in the case that the leaf $\mathcal{L}$ has non-trivial holonomy, we know that the holonomy group is
finite and so the same Reeb result tells us that there still exists a saturated open set $U$, which is a fibration in
discs of dimension complementary to the dimension of $\mathcal{L}$, whose structure group is the holonomy group.
That is, the leaves in this open set are coverings of our original leaf and this open subset of $\mathcal{L}$ gives $M/\mathcal{F}$
a chart of complex orbifold around this point.
This can be contrasted with the original statement and proof of Edwards, Millet and Sullivan, which are more general, but at the same time lose
the peculiarities of Kähler geometry, furthermore Dr. Verjovsky and I place greater emphasis on the structure of the
leaf space (see \cite{EMS}).

It is important to observe that it is possible to find foliations on compact real analytic and even algebraic manifolds, where
all the leaves of the foliation are closed curves (circles) whose lengths are not uniformly bounded, see the example in \cite{E-V} and \cite{Epstein2}.
Something important to highlight from this example is that the codimension is large enough
to guarantee that the volume (length) is not uniformly bounded and it also happens that the length function of the leaves
is not locally bounded.

The aforementioned puts us in the context of Beauville's theorem (see \cite{Beuville}).
\begin{theorem}[Beauville]\label{Beauville}
        Let $M$ be a compact Kähler variety such that there exists a holomorphic decomposition of its tangent bundle
        \[
        TM=\bigoplus_{i\in I}\mathcal{F}_i\hspace{0.2cm}\text{such that each}\hspace{0.2cm}\bigoplus_{i\in J}\mathcal{F}_i\,,\,J\subset I\hspace{0.2cm}\text{is involutive},
        \]
        then the universal cover of $M$ is isomorphic to a product
        \[
        \widetilde{M}\cong\prod_{i\in I}U_i\hspace{0.2cm}\text{in such a way that this induces an isomorphism}\hspace{0.2cm}T\widetilde{M}\cong\bigoplus_{i\in I}\widetilde{\mathcal{F}_i}
      \]
      where $\widetilde{\mathcal{F}_{i}}$ projects onto $\mathcal{F}_{i}$ under the covering map.
\end{theorem}
Recently, Druel, Pereira, Pym and Touzet proved a version of this theorem in the context that we previously exposed,
but with the particular focus on Poisson varieties, see \cite{DPPT}.
\begin{theorem}[Druel, Pereira, Pym and Touzet]\label{DPPT}
        Suppose that $M$ is a compact Kähler manifold such that its tangent bundle splits $TM=\mathfrak{F}\oplus\mathfrak{G}$, where
        the subbundles $\mathfrak{F}$ and $\mathfrak{G}$ are involutive. If $\mathfrak{F}$ has a compact leaf $L$ with finite holonomy,
        then $\widetilde{M}$ is biholomorphic to a product of manifolds $N\times P$ whose tangent bundles are isomorphic
        to $\mathfrak{F}$ and $\mathfrak{G}$ respectively.
\end{theorem}
\noindent As an observation, it can be seen from the previous result that in a connected Kähler manifold, the existence
of a compact leaf with finite holonomy implies that all leaves are. We were also able to prove this using
Theorem \ref{Bishop2}. This result was already known, but our proof is different from the one given in the
original paper \cite{Pereira}.
We believe that it is possible to prove the proposition of Druel, Pereira, Pym and Touzet using limits of the leaves
in a similar way as previously exposed. More on this will be discussed in the section on \textit{Problems to be solved}.

It is clear from what has been shown here that there is an important link between the structure of an analytic space and its volume.
In the case of foliations, we can extend this notion to the volumes of its leaves, and Bishop's results
provide us with a bridge between geometry and analysis, so we propose to study deeper connections between these
two areas using the tools previously exposed as well as other methods from modern complex geometry.
As has been alluded to earlier, holomorphic foliations in Kähler type manifolds are of particular interest in this regard
and therefore it is in this context that we think the expansion of our research could be most fruitful in the
pursuit of new results.



\section{Fano-Poisson Varieties}
\noindent Inspired by the proof of Theorem \ref{DPPT} from the recent paper \cite{DPPT}, we believe that there exist
interesting relationships between the volume function of the leaves defined by the natural foliation of a \emph{Poisson} variety
(Theorem \ref{weins}) and the structure of a Fano-Poisson variety.

\subsection{Poisson Varieties, Weinstein's Theorem and its Natural Foliation}
\noindent First, we must recall that a complex manifold $M$ is Poisson if there exists a bilinear operation on the ring of germs of
holomorphic functions on $M$, which is known by the name \emph{Poisson bracket}.
We will denote by $\mathcal{O}_M:=\mathcal{H}_M/\sim$ the ring of germs of holomorphic functions, where $f_1\sim f_2$ if
$f_1=f_2|_U$ for some open set $U\subset\con$. Thus, a Poisson bracket is a bilinear function
\[
\{\cdot,\cdot\}:\mathcal{O}_M\times\mathcal{O}_M\rightarrow\mathcal{O}_M,
\]
that satisfies the following properties
\begin{enumerate}
\item $\{f,g\} = -\{g,f\}$
\item $\{f,gh\}=\{f,g\}h + g\{f,h\}$
\item $\{f,\{g,h\}\}+\{g,\{h,f\}\} + \{h,\{ f,g\}\}=0$.
\end{enumerate}
\noindent It is observed that a germ of a fixed function $H\in\mathcal{O}_M$ definitionnes a vector field given by $\xi_H(\cdot)=\{H,\cdot\}$.
We call this vector field the \emph{Hamiltonian} field defined by $H$. Using the notation of \emph{multivector} fields,
an alternative definition of the Poisson bracket is the following: denoting the holomorphic vector fields as $\mathcal{T}M=H^0(M,TM)$,
then the space of \emph{$p$-vectors} is defined by
\[
        \Lambda^{p}(\mathcal{T}M):=\{\mathcal{O}_M\times\dots\times\mathcal{O}_M\rightarrow\mathcal{O}_M\,\vert\,\text{anti-symmetric}\}.
\]
Then, a Poisson bracket is a \emph{bivector} field, which we can definitionne via the pairing $\langle\cdot,\cdot\rangle$
between the $p$-vector fields and the space of \emph{holomorphic $p$-differential forms} $\Omega^{p}(M)$ by
\[
        \pi\in H^0(M,\Lambda^2(\mathcal{T}M))\,\text{ then }\,\{f,g\}=\langle \pi,df\wedge dg\rangle.
\]
Therefore, generalizing this we have the following map defined by a Poisson bracket $\pi$
\[
        \pi^{\#}:\Omega^1_M\rightarrow\mathcal{T}M,\hspace{0.2cm}\pi^{\#}(\alpha):=\iota_{\alpha}(\pi):=\langle\pi,\alpha\wedge\cdot\rangle.
\]
\noindent With this, the rank of $\pi$ at a point $p\in M$ is defined as the positive integer $r\in \zah^{+}$ that definitionnes the dimension of the
maximum space where $\pi^{\#}_p$ is non-degenerate, i.e., it is the dimension of the largest space such that $\pi^{\#}$ is a
bijection. If $\pi^{\#}$ is non-degenerate, its rank is $2n=\dim(M)$, then $\pi^{-1}$, the inverse of $\pi^{\#}$, definitionnes a symplectic form
on $M$. Weinstein's splitting theorem definitionnes a natural foliation on a Poisson variety $(M,\pi)$, but first
we recall that a holomorphic map between two Poisson varieties $(M_1,\{\cdot,\cdot\}_1)$ and $(M_2,\{\cdot,\cdot\}_2)$ is a Poisson morphism
$\phi:M_1\rightarrow M_2$ if $\{f,g\}_1\circ\phi=\{f\circ\phi,g\circ\phi\}_2$.
\begin{theorem}[Weinstein]\label{weins}
        Let $(M,\pi)$ be a holomorphic Poisson manifold of real dimension $2n$. Suppose that $\pi$ has rank $2r$ at a point $x\in M$,
        then there exists a neighborhood $U$ of $x$ such that $U$ is isomorphic in the Poisson sense to a product $S\times P$, where $S$ is
        symplectic with coordinates $(p_i,q_i)_{i=1}^r$, and $(P,\tilde{\pi})$ is a Poisson manifold of rank zero at $x$
        with coordinates $z=(z_j)_{j=1}^{2n-2r}$
        \[
                \pi=\sum_{i=1}^r \partial{p_i}\wedge\partial{q_i}+\sum_{1\leq j\leq k\leq 2n-2r} f^{jk}(z)\partial{z_j}\wedge\partial{z_k}.
        \]
\end{theorem}
\noindent We observe from the previous theorem that $f^{jk}(x)=0$. This theorem clearly definitionnes a natural foliation on $(M,\pi)$ by
symplectic leaves. However, not all leaves have the same dimension, so it is necessary to note that the definition of foliation
can be expanded to this more general context, i.e., a foliation is simply an involutive $\mathcal{O}_M$-submodule of $\mathcal{T}M$.
Now, this definitionnes a filtration $X_0\subset X_2\subset X_4\subset\dots\subset M$, where $X_{2k}=\{x\in M\,|\,\textrm{rank}(\pi_x)\leq 2k\}$,
if we denote the symplectic leaves of $(M,\pi)$ as $\mathcal{L}$, then we can also think of $X_{2k}$ as
$$
X_{2k}=\bigcup_{\dim(\mathcal{L})\leq 2k}\mathcal{L}.
$$

\subsection{Fano-Poisson Varieties from the Perspective of Kähler Geometry}
\noindent With this established, we will present a theorem posed in 1993 by Bondal for Fano-Poisson varieties that we hope to shed new light on using the volume function. A variety $M$ is Fano if it satisfies the following
(see \cite{S-Yau} and \cite{ZB}):
\begin{itemize}
        \item $M$ admits a Kähler-Einstein metric, i.e., if we definitionne the metric via its Kähler symplectic form
        $\omega$, then
                $$\textrm{Ric}_{\omega}=\lambda\omega\hspace{0.2cm}\lambda\in\re,$$
        where in coordinates we can calculate the Ricci curvature for the metric
        \hbox{$h=g-i\omega=\sum h_{i\overline{j}}\,dz_i\otimes d\overline{z}_j$}
        as
$$\textrm{Ric}_{\omega}=\frac{i}{2\pi}\sum_{ij}R_{i\overline{j}}\,dz_i\wedge d\overline{z}_j=\frac{-i}{2\pi}\partial\overline{\partial}\log(\det(h_{k\overline{l}})).$$
        \item The cohomology class defined by the Ricci curvature (first Chern class) is positive
        $$
        [\textrm{Ric}_{\omega}]=c_1(M)>0.
        $$
\end{itemize}
\noindent It should be mentioned that usually in the literature a Fano variety is defined as a complete algebraic variety $X$
in the sense that every projection $X\times Y\rightarrow Y$ is proper for every algebraic variety $Y$, such that its anti-canonical divisor/sheaf
$K^{*}_{X}$ is ample, thus every Fano variety is \emph{projective}. The discrepancy between the definition we gave and the one widely used in
the literature comes from Yau's proof of the Calabi theorem (see \cite{S-Yau}). A
\emph{Fano-Poisson} variety is then a complex manifold with these two structures. By the Bonnet-Myers theorem, such manifolds
are compact Kähler (see \cite{Myers}).
\begin{theorem}[Bondal]\label{Bondal}
  Let $M$ be a Fano-Poisson variety, if $X_{2k}$ is the union of the symplectic leaves defined by Weinstein's theorem (Theorem \ref{weins})
  of real dimension (or rank) less than $2k$, then $X_{2k}$ has a component of dimension greater than $2k$ (see \cite{Bondal}).
\end{theorem}
\noindent We do not know if it is true or false or if it is possible to prove it with what we propose, but we believe that as an
investigation into the relationship between complex-differential geometry and algebraic geometry, what we propose is interesting.
We believe that this connection is relevant to the geometric approach that we propose, inspired by previous works
of Yau \cite{S-Yau} and especially Donaldson and Sun \cite{D-SS}. In particular, Donaldson and Sun show results of a very similar nature
to Bishop's results (Theorem \ref{Bishop2}) in the generalized context of Gromov-Hausdorff limits in
Kähler manifolds, with particular application to Fano varieties.

\section{Speculations and Problems to be Solved}
\begin{itemize}
%\textcolor{red}{\centerline{Problem to be solved, hypothesis:}}
        \item In the case of Beauville's theorem \ref{Beauville}, we first intend to prove the case where
        the leaves are compact, using Bishop's methods. Thus, we would obtain an alternative proof of
        Theorem \ref{DPPT}.

        \item We believe it is possible to remove the assumption of compact leaves by only assuming that the
        leaves have locally bounded volume, i.e., that the leaves have finite volume and at any point there exists
        a neighborhood such that the foliation in that neighborhood consists of leaves with uniformly bounded volume. Under this
        assumption, we can guarantee that the leaf space is a complex analytic space, and the foliation is
        locally a fibration (see \cite{A-V}).

        \item Under these assumptions, we believe it is possible to definitionne a biholomorphism between the universal covers
        of the leaves by means of liftings of curves and work with the holonomy groupoid of the foliation
        similar to what was done in \cite{DPPT}. Furthermore, we believe it is possible to extract metric information from the
        leaves (or their cover) if we lift geodesics instead of arbitrary curves, where we simply use the
        Riemannian metric on $M/\mathfrak{F}$ inherited from the original Kähler metric.

        \item In the case of foliations with bounded volume, we speculate that the phenomenon of semi-continuity that was exposed in the case
        of volumes, occurs similarly at the level of homotopy groups and that it may possibly allow us to understand more about the
        analytic structure of the universal cover of the leaves, which in the case of Kähler manifolds is the same for all of them
        if Beauville's hypothesis is valid. That is, in the limit of a sequence of leaves with bounded volume, their universal
        cover is the same and the fundamental group of the limit has as a subgroup that of the leaves approaching it,
        it may be necessary to ask that these groups be finite.

        \item Provide a homological/cohomological meaning to the discrete semi-continuity in Theorem \ref{EMS}.

        \item If we assume that the space $M$ or the leaf space $M/\mathfrak{F}$ have additional structure (Poisson, Fano, Stein, projective)
        what kind of behavior can the volume functions of a variety have?

%\textcolor{red}{\centerline{We speculate:}}
        \item If $\mathfrak{F}$ is a foliation with an isolated singularity at the origin of $B$ whose leaves are uniformized by the disk $\Delta$,
        then there exists an incomplete leaf and a geodesic on it whose limit is $0$.

%\textcolor{red}{\centerline{We propose:}}
        \item Study the growth of the volume of the sets $X_{2k}\cap B_{R}(x)$ where $B_R(x)$ is the metric ball
        of radius $R$ centered at a point $x\in X_{2k}$ when $R\rightarrow\infty$. If the set $X_{2k}$ is
        of dimension greater than $2k$, we expect a volume growth behavior greater than $\mathcal{O}(R^{2k}).$

        \item Look for cases in which there exist descriptions of the sets $X_{2k}$ that allow us to describe these sets
        as Gromov-Hausdorff limits and make use of bounds given in \cite{D-SS} to determine the rate of growth
        of the volume of the symplectic leaves.

        \item Determine if the non-collapsing volume condition given in \cite{D-SS} and similar analytic estimates
        allow us to calculate the volume of subvarieties in Fano manifolds. If this is possible, find
        what kind of properties the volume function has on a foliation of a Fano variety.

        \item Make a contrast between the geometric aspects found and the proofs of Theorem \ref{Bondal} in the
        cases already proven as in \cite{Gua-Pym}.

\end{itemize}

\subsection{Vanishing Cycles and Leaf Limits}
\noindent Essentially, I believe that if a leaf of the foliation is approximated by others, the topology should not grow, and this reflects that there are vanishing "loops" cycles, meaning that generators of the fundamental group should disappear in the limit as they become small. Now, in terms of what has been previously investigated, that is, the volume. This means that we have an inverse effect to what happens with the volumes of the leaves, that is, if the volume grows and takes a jump, then the fundamental group becomes small and we are in the presence of a vanishing cycle, this should be seen for example in fundamental regions of hyperbolic curves for example.
\subsection{Lelong Numbers}
\noindent From the candidacy exam, I believe that the idea of seeing the growth rate of volumes is a powerful tool and that directly relates us to the Lelong numbers. So it is necessary to see if there is any topological (homological or cohomological) interpretation of these. Below is a summary of the article "Propiétés Métriques Des Ensembles Analytiques Complexes".
\subsubsection{Introduction}
\begin{quote}
The notion of "area" of complex analytic sets is now clearly established. It derives from the existence of the integral of a differential form over such a set. This definitionnes the basis of a method for studying the metric properties of complex analytic sets. In turn, in the study of these, a particular class of linear operators appears: currents or generalized differential forms, which are at once positive and closed. In more abstract terms, one seeks a functor from the category of complex analytic sets to the category of such currents. For the search of this functor, it seems that a new and effective viewpoint is provided by the theorems of the extension of analytic sets such as the Remmert-Stein theorem and the proof of W. Stoll's theorem, proved by E. Bishop. On the other hand, a closed family of analytic sets is a normal family, which is compared with P. Montel's classical theorem for holomorphic functions.
\end{quote}

\noindent It then proceeds to definitionne what the notion of "area" is, which applies to objects that are not varieties, and that Lelong's exposition that follows comes from the study of real analytic and semi-analytic sets done by M. Herrera, together with this the mathematical tool that is used for this is classical (measures and distributions).

Recall that a set $A$ is a complex analytic subset of the analytic manifold $M$ if it is closed and every point $x\in A$ possesses a neighborhood $U_x$ such that $A\cap U_x=V(f_1,\dots,f_k)$ where $V(F)$ denotes the zero set of the family of functions in $F$ and $f_j:U_x\rightarrow\mathbb{C}$. An analytic set in general does not have a differential manifold structure. So it is necessary to definitionne an \emph{ordinary point} of $A$ as a point $x\in A$ such that in a chart $(U_x,\varphi)$ of $x$, $U_x\cap A$ is biholomorphic to a neighborhood of a linear subspace $L^{k}\subset\mathbb{C}^n$ by means of a restriction of $\varphi$, the complex dimension of $L^{k}$, $k$ is the dimension of $A$ at $x$, we will denote the set of ordinary points by $A'$. The ordinary points form an analytic subset of $A$ of dimension $k'\leq k$. If $A$ is of dimension $k$ at all its points, we will say that $A$ is of pure dimension $k$.

Since an analytic set is not a manifold, it is not evident that certain analytic operations can be performed on it, for example, integration over itself.

Let $A$ be a pure dimension $k$ analytic subset defined in a domain $U\subset\mathbb{C}^n$. We denote by $\Omega_{(p,p)}(U)$ the space of differential forms with $C^{\infty}$ coefficients of compact support of type $(p,p)$. We definitionne the following linear operator

\[
    t(\phi)=\int_{A}\phi
\]

If $A$ is an analytic set, the analytic set of non-ordinary points of $A$ is $A\setminus A'\subset U\setminus A'=U'$ so we will say that $U'$ is the (open in $\mathbb{C}^n$) set of non-ordinary points of $U$, then the problem of finding $t(\phi)$ is the problem of extending the operator
\[
    t_0(\phi)=\int_{A}\phi
\]
from the space $\Omega_{(p,p)}(U)$ (where it is well-defined in the classical sense) to $\Omega_{(p,p)}(U)$.
\subsubsection{Positive Currents and Closed Positive Currents}
\label{sec:org61c3412}
\begin{quote}
The operator $t_0(\phi)$ is a current in the sense of de Rham, also known as a \emph{generalized differential form} or differential form with coefficients that are distributions. The space is described as the space of expressions of the type
\[
    t=\sum_{i_1,\dots i_{n-p},j_1,\dots j_{n-p}}t_{i_1\dots i_{n-k}j_1\dots j_{n-k}} dz_{i_1}\wedge\dots\wedge dz_{i_{n-k}}\wedge d\overline{z}_{j_1}\wedge\dots\wedge d\overline{z}_{j_{n-k}}.
\]
\end{quote}
Moreover, if $\tau_n$ denotes the invariant or "volume element" form on $\mathbb{C}^n$, that is $\tau_n=\omega_n^n/n!$, where
\[
    \omega_n=\frac{i}{2}\sum_{j=1}^n dz_j\wedge d\overline{z}_j,
\]
the Kähler form on $\mathbb{C}^n$. The mapping $t_{i_1,\dots i_{n-p},j_1,\dots j_{n-p}}\rightarrow T_{i_1,\dots i_{n-p},j_1,\dots j_{n-p}}\tau_n$ canonically associates the coefficients of $t$ with distributions $T_{i_1,\dots i_{n-p},j_1,\dots j_{n-p}}$. On the other hand, to a subspace $L^k$,
\noindent it is possible to canonically associate a differential form $\omega(L^k)$ as follows: consider the transformation $\mathbb{C}^n\rightarrow L^k$, which is defined as $z=(z_1,\dots,z_n)\mapsto z'=(z_1,\dots,z_k)$, where $z_{j+k}\mapsto 0$ for $j\in\{1,\dots,n-k\}$. We definitionne
\[
    \tau(L^k):=\frac{\omega_n^k}{k!}(dz')=(\frac{i}{2})^k(-1)^{k(k-1)/2}\sum \alpha(s)\overline{\alpha(t)}dz_{s_1}\wedge\dots\wedge dz_{s_k}\wedge d\overline{z}_{t_1}\wedge \dots\wedge d\overline{z}_{t_k}.
\]
Where $\alpha(s)=\Vert(a^{1}_{j},\dots,a^{k}_{j})\Vert$ $j\in(s)=\{A\subset\{1,\dots,n\}\,\vert\, |A|=s\}$ with $1\leq s\leq k$, that is, we see $L^k$ as
\[
    L^k=\{z_j\,\vert\,z_j=\sum a^{s}_{j}z'_s,\quad 1\leq j\leq n,\quad 1\leq s\leq p \}
\]
Thus it can be verified that $L\mapsto\omega(L^k)$ is an injective mapping from the space of subspaces of dimension $k$ (Grassmannian) into the space of differential forms. It can be verified that $\omega(L^k)^{*}=\omega(L^{n-k})$ where $L^{n-k}$ is the orthogonal complement of $L^k$.
\begin{quote}
\begin{definition}
    A current $t$ is positive if it satisfies
    \begin{itemize}
        \item[a)] It is of type $(q,q)$ with $0\leq q\leq n$
        \item[b)] For every $L^{n-q}$, $\sigma[t,L^{n-q}]=t\wedge\omega(L^{n-q})$ is a positive distribution (functional)
    \end{itemize}
    This ensures that $\sigma[t,L^{n-q}]$ is a positive measure.
\end{definition}
\end{quote}
\begin{prop}
The following propositions hold for positive currents:
    \begin{itemize}
        \item[1] The application $t\mapsto\sigma[t,L^{n-q}]$ is injective and on the complex Grassmannian, for any given point $L^{n-q}_{0}$, there exists a neighborhood of this point where it is possible to find points $L^{n-q}_s,\quad s\in\{1,\dots,{n\choose k}\}$, such that the coefficients $t_{I,J}$ of $\omega(L^{n-q}_{0})$ are expressed as linear combinations of $\sigma_s=\sigma[t,L^{n-q}]$. We will call a system that satisfies this \emph{regular}.

        \item[2] As a consequence of the previous point, for a positive current, the $T_{I,J}$ associated with $t_{I,J}$ are complex measures, so $t$ extends to the space $C^{0}_{(q,q)}$ of differential forms with continuous coefficients of type $(q,q)$.

        \item[3] Let $T_{>0}^{p}$ be the cone of positive currents of type $(p,p)$ (The set of positive currents is closed under multiplication by a positive constant). Let $\Phi^{p}_{>0}$ be the cone of positive currents that are representable by forms with continuous coefficients (we will say that positive currents of type $(p,p)$ are of degree $p$ or of dim $n-p$). If $t\in T^{p}_{>0}$ and $\phi\in\Phi^{1}_{>0}$, then $t_1=t\wedge\phi\in T^{1}_{>0}$. In particular the forms $\omega_n$, $\omega^{k}_n$ and $\omega(L^k)$ are positive, as is $i\partial\overline{\partial}\eta$ if $\eta$ is a plurisubharmonic function.

        \item[4] If $t\in T^{n-p}_{>0}$ is a positive current of degree $n-p$, then
        \[
            \sigma=\frac{1}{p!}t\wedge\omega_n^{p}
            \quad\text{and}\quad
            \nu=\frac{1}{\pi^p}t\wedge\Big(\frac{i}{2}\partial\overline{\partial}\log\Vert z-\zeta\Vert\Big)^{p},
        \]
        are positive measures, where the right-hand side of $\nu$ is the metric of the projective space at the point $\zeta$, normally denoted $\omega_{FS}$. It is verified that in an open set $U$
        \[
            \Vert\sigma\Vert_{U}=\int_{U}\sigma,
        \]
        is a norm equivalent to the operator norm on $T^p_{>0}$ defined by
        \[
            \Vert t\Vert_U=\sup\{|t(\phi)|\,\vert\,\phi\in \Omega_{(p,p)(U)}\quad\Vert\phi\Vert=1\}.
        \]
        Moreover $\Vert t\Vert_U=\sup\{\Vert t\wedge\omega(L^{p})\Vert_U\,\vert\,L^{p}\text{ in a regular system}\}$
        \item[5] If $t$ is closed, that is
        \[
            d(t)[\phi]:=t(d\phi)=0\quad\forall\phi\in\Omega_{*}(U),
        \]
        then the following property holds: Let
        \[
            B_1=B(z,r_1)\quad B_2=B(z,r_2)\quad 0<r_1<r_2.
        \]
        Concentric metric balls with center at $z$. Then we have
        \[
            0<\int_{B_2\setminus B_1}\nu= \frac{p!}{\pi^{p}}(\frac{\sigma(r_1)}{r^{2p}_2}-\frac{\sigma(r_2)}{r^{2p}_1}).
        \]
        where $\sigma$ and $\nu$ are as defined previously. Thus it turns out that for a positive current $t$ of degree $n-p$, the function $\sigma(r)r^{-2p}$ is an increasing function of $r$ and the limit
        \[
            \nu(z,0)=\lim_{r\rightarrow 0}\frac{p!}{\pi^{p}}\sigma(r)r^{-2p},
        \]
        exists and extends the measure $\nu$ from $B\setminus\{z\}$ to the point including $z$ with a positive point measure at $z$.
    \end{itemize}
\end{prop}


\subsubsection{Integration Currents}
\label{sec:org9982aa5}
On a complex analytic set $A\subset U\subset\mathbb{C}^n$, we have the following theorem:

\begin{theorem}
The operator
\[
    t(\phi)=\int_A\phi
\]
defined by a simple extension of $\Omega_{(p,p)}(U\setminus A')$ to the space $\Omega_{(p,p)}(U)$, exists and is unique and has three properties:
\begin{itemize}
    \item[a] It is the unique extension of $t_0(\phi)$ that has zero norm on $A'$ (simple extension).
    \item[b] It is closed.
    \item[c] It is positive definitionnite.
\end{itemize}
\end{theorem}

\noindent \textbf{IMPORTANT:} The proof is based on the "Stokes theorem for differential structures", see "LELONG P. - Intégration sur un ensemble analytique complexe", which basically says that if a closed current has a zero-order continuation in $U\setminus E$ where $E$ is a closed set, due to its extension from $U\setminus E$ to $U$, it is necessary and sufficient that there exists a family of $C^{1}(U)$ functions, say $\alpha_s(x)$ such that $\alpha_s(x)\geq0$, $\alpha_s(x)=1$ in a neighborhood $W_s$ of $E$ and whose support lies in a neighborhood $V_s$ of $E$, where $V_s\rightarrow E$ as $s\rightarrow 0$, and we also want
\[
    \lim_{s\rightarrow 0}t\wedge d\alpha_s=0.
\]
This condition is local and it is sufficient to construct such nuclei relative to compacts of E. In particular, if $E$ is the $C^1$ image of $\mathbb{R}^m$ of real dimension $m$, it is sufficient to prove
\[
    \lim_{r\rightarrow 0}1/r\Vert t\Vert^{r}_K=0,
\]
for every compact $K$, where we measure the norm as usual, but in
\[
 U_r=\{x\in K\,\,\vert\,d_H(x,E)<r\}
\]
The following result allows simplifying the proof:
\begin{prop}
Let $A$ be a pure dimension $k$ analytic set containing the origin and let $\eta$ be the set of $L^{n-k}$ spaces that touch $A$ in such a way that the origin is NOT an isolated point of the intersection. Then
\begin{enumerate}
\item $\eta$ is identified with an analytic subset (in the algebra of the Grassmannian)

\item There exists an orthogonal coordinate system at the origin such that any $(n-k)$-dimensional subspace not belonging to $\eta$

\item There exists a polycylinder relative to such a coordinate system
\[
    P=\{z\,|\,|z_j|<a_j,\,j\in\{1,\dots,n\}\,a_j>0\},
\]
such that the projections of $A\cap P$ onto the $\mathbb{C}^{k}$ coordinate spaces are proper
\end{enumerate}
\end{prop}
This statement makes evident the existence of a ball $B=\{\Vert z\Vert<r\}$ such that
\[
        \sigma(r)\leq kr^{2k}
\]
is an increase of the norm of the operator $t_0$ and allows the application of the general Stokes theorem.

\subsubsection{Bounded Families of Analytic Sets}
\label{sec:org50c0940}
Now consider the following mapping
\[
    A\mapsto t(A)\in T^p_{>0}=\{t\,|\,\text{ positive current}\},
\]
\noindent which associates to an analytic set $A$ of pure dimension $k$, a closed positive current $t$.

\textbf{Families of positive currents in a domain $D$ give rise to the same properties as families of positive measures}

\begin{definition}
We will say that $t_n\rightarrow t$ \emph{weakly} if $t_n(\varphi)$ has a limit for every $\varphi\in\Omega_{(p,p)}(D)$
\end{definition}

\begin{enumerate}
\item If $\{t_n\}$ converges weakly, then $t=\lim t_n\in T^p_{>0}$. Moreover, the norms $\Vert t_n\Vert$ are \textbf{BOUNDED} on every compact set of $D$.
\item If the norms $\Vert t\Vert$ are bounded for every compact set, where $t\in F$, then we will say that the family $F$ is \emph{normal}. This means that every sequence $\{t_n\}\subset F$ has a weakly convergent subsequence.
Consider the products
\end{enumerate}
\[
    t_k\wedge\omega(L^{n-p}_s)\quad\text{where }L^{n-p}_s\text{ is a regular system}.
\]
\begin{enumerate}
\item Let $F$ be a family of closed positive currents of dimension $p$ in $D$. Then for every compact $K\subset D$, the numbers
\[
        \nu(\zeta,0),\nu(\zeta,r)=p!\pi^{-p}\sigma(\zeta,r)r^{-2p},
   \]
\noindent are bounded. Where $\sigma(\zeta,r)$ is the measure $\sigma$ on the ball $\{\Vert z-\zeta\Vert<r\}$ with $\zeta\in K$.
\item If $t_n\rightarrow t$ weakly, and $z_m\rightarrow z$, then we have
\end{enumerate}
\begin{equation}
        \nu_t(z,0)\geq\limsup\nu_{t_k}(z_k,0).
\end{equation}
\begin{enumerate}
\item In particular, $\nu(z,0)$ is an upper semicontinuous function.
\end{enumerate}

\begin{definition}
A family $\{A_i\,|\,i\in I\}$ of analytic sets of pure dimension $k$ is locally bounded in $D$ if the currents $\{t_i=t[A_i]\}$ form a locally bounded family in $D$.
\end{definition}

\subsubsection{Properties of $\nu(z,0)$ for Analytic Sets}
\label{sec:org81bac70}
For divisors $\{f=0\}$, it is easy to see that if
\[
    f=P_q+P_{q+1}+\dots\quad\P_j\text{ homogeneous polynomial of degree j},
\]
\noindent in a neighborhood around the origin, then
\[
    \nu(0,0)=q.
\]
On the other hand, an elementary proof gives us the following result: if $z\in A$ is an ordinary point, then $\nu(z,0)=1$. Thus it is easy to prove that $\nu(z,0)\geq1$ for all $x\in A$. Thie proved that $\nu(z,0)\in\mathbb{Z}$.

\begin{prop}[Thie]\label{thie}
Let $z\in A$, where $A$ is a pure dimension $k$ analytic set. Then
\[
    \nu(z,0)=\sum^{b}_{s=1}m_{s} m_{s}',
\]
\noindent where $s\in\{1,\dots,b\}$ is the index relative to the decomposition of the tangent cone at $z$, $T$Here is the continuation:

into components $T_1, \dots T_b$, where $m_s'$ is the degree of $T_s$ and $m_s$ is the degree of the holomorphic volume form of $A$ over $T_s$ in a neighborhood of the origin.
\end{prop}

\textbf{Consequences:}
if $\sigma(z,r)$ is the "area" of $A$ in the ball $B_r(z)=\{\|z-x\|\leq r\}$ and $l(x,r)$ is the $(2k-1)$-dimensional volume of $A\cap\partial B_r(z)$, now if $\tau_{2k}$ and $\omega_{2k-1}$ are the volume forms of the open ball and its boundary respectively, then
\begin{align*}
\sigma(z,r)\geq\nu(z,0)\tau_{2k}r^{2k}\geq\tau_{2k}r^{2k}\\
l(z,r)\geq\nu(z,0)\omega_{2k-1}r^{2k-1}\geq\omega_{2k-1}r^{2k-1}
\end{align*}

\noindent In particular if $A$ is of dimension one, $l(z,r)\geq2\pi r$.

Thus, the class of positive currents that are currents arising from a complex analytic set $t=t[A]$, is distinguished by the following properties:
\begin{enumerate}
\item $t\in T^p_{>0}$ obvi
\item $t$ is closed (operator)
\item The number $\nu(z,0)$ with $z\in \mathrm{supp}(t)$ (support) has a lower bound $c\in\mathbb{R}^{+}$.
\item The number $\nu(z,0)$ with $z\in \mathrm{supp}(t)$ is an integer.
\end{enumerate}

Naturally the question arises: Do there exist positive currents that have the first three properties but do not come from analytic sets? We will call these currents of class $\Lambda$.

\begin{prop}
If $\{t_n\}\subset\Lambda$ and $t_n\rightarrow t$ weakly, then $t\in\Lambda$ and $\lim \mathrm{supp}(t_n)=\mathrm{supp}(t)$.
\end{prop}

For $t\in\Lambda$, the $(2k+1)$-dimensional Hausdorff measure of $\mathrm{supp}(t)$ is zero, i.e. the Hausdorff dimension of $\mathrm{supp}(t)$ is less than $2k+1$. It turns out that the set of $L^{n-k}$ spaces that intersect $\mathrm{supp}(t)$ at a point and such that their intersection contains a continuum, is of first category, i.e. it is a countable union of nowhere dense sets.

\begin{prop}[Bishop]
Let $\{A_n\}_{n\in\mathbb{N}}$ be a sequence of pure dimension $k$ analytic varieties in a domain $D$ such that $\{t(A_n)\}$ form a bounded family on every compact set, and further suppose that $A_n\rightarrow A$ in the sense of the Hausdorff metric, then $A$ is an analytic set.
\end{prop}

It should be noted that there exist coordinate axes such that $\{z_1=0,\cdots,z_k=0\}\cap A$ is a disconnected set. A point of $A$ has a hollow tubular neighborhood, of type $S\times bT$ such that $(S\times bT)\cap A=\emptyset$, which implies that $(S\times bT)\cap A_n=\emptyset$ for all $n>N$. It follows that almost all sections of the $A_n$ by $S\times\{x\in T\}$ are of dimension zero, i.e. the "projection" maps $\pi:A_n\rightarrow S$ are proper. These maps have a degree $\lambda_n$ which is evidently bounded. Therefore if $z_0\in(S\times T)\setminus A$, we define $f_n(z)=\pi[g(z)-g(z_{i,n})]$ where the points $\{z_{i,n}\}$ are defined as
\[
\{z_{i,n}\}=\pi^{-1}[\pi(z)]\cap A_n,
\]
and where $g(z)$ is a holomorphic function on $S\times T$ with $g(z_0)\neq0$. The functions $f_n$ are holomorphic on $S\times T$ and converge uniformly to $f(z)$ (the corresponding one to $A$). Thus $A\subset\{z\,|\,f(z)=0\}$ and $f(z_0)\neq0$. Therefore $A$ is an analytic set.

\begin{theorem}
Locally bounded families $F$ of complex analytic sets are normal.
\end{theorem}

Indeed, every sequence $\{A_n\}\subset F$ has a subsequence $\{A_m\}$ such that $t_m=t(A_m)$ converges. Let $t=\lim t_m$, then
\[
    A_n=\mathrm{supp}[t(A_m)]\rightarrow\mathrm{supp}t=A
\]
and as we know $A$ is an analytic set.

\begin{enumerate}
\item Consequences
\label{sec:org6dba31e}
\begin{enumerate}
\item \textbf{The Remmert-Stein Theorem:} If $B$ is an analytic set of $D$ of dimension $\dim(B)\leq k-1$ and $A$ is a pure dimension $k$ analytic set in $D\setminus B$, then $\overline{A}$ is an analytic set in $D$.
\item \textbf{Stoll's Theorem:} If $B$ is an analytic set of $D$ and $A$ is a pure dimension $k$ analytic set in $D\setminus B$ and moreover $\|t(A)\|_{D\setminus B}\leq\infty$ then $\overline{A}$ is an analytic set in $D$.
\end{enumerate}
\end{enumerate}

\end{document}
